% \iffalse meta-comment
% A Package for recording errata
% Copyright (c) 2006 Michael Kohlhase, all rights reserved
% Copyright (c) 2015 Bendik Rønning Opstad
%
% This file is distributed under the terms of the LaTeX Project Public
% License from CTAN archives in directory  macros/latex/base/lppl.txt.
% Either version 1.0 or, at your option, any later version.
%
% The development version of this file can be found at
% https://github.com/KWARC/LaTeX-errata
%
% \paragraph{Acknowledgements} Thanks to Johan Schroll-Fleischer for comments and bug
% reports.
%
% \fi
%
% \iffalse
%<*driver>
\documentclass[twoside]{ltxdoc}
\makeatletter
\def\SpecialMainIndex#1{%
\@bsphack
\immediate\write\@auxout{%
\global\noexpand\expandafter\let\noexpand\csname MAIN:\noexpand\string\string#1\endcsname\noexpand\@empty}%
\SpecialIndex@{#1}{\encapchar main}\@esphack}
\def\SpecialIndex#1{%
\@bsphack
   \expandafter\ifx\csname MAIN:\string#1\endcsname\@empty
   \special@index{\expandafter\@gobble
                                      \string#1\actualchar
      \string\verb\quotechar*\verbatimchar\string#1\verbatimchar}%
   \fi
    \@esphack}
\makeatother
\DoNotIndex{\def,\long,\edef,\xdef,\gdef,\let,\global}
\DoNotIndex{\begin,\AtEndDocument,\newcommand,\newcounter,\stepcounter}
\DoNotIndex{\immediate,\openout,\closeout,\message,\typeout}
\DoNotIndex{\pageref,\parindent}
\DoNotIndex{\section,\scshape,\arabic,\\}
\usepackage{textcomp,url,a4}
\usepackage{hypdoc}
\hypersetup{%
  linkcolor=black,
 }
\usepackage[T1]{fontenc}
\usepackage[utf8]{inputenc}
\usepackage[margin=1.4in,verbose=false]{geometry}
\input{preamble}
\input{erratadocutil}
\usepackage[show]{ed}
\usepackage[show]{errata}
%\usepackage[colorlinks=true,linkcolor=black,linktocpage=true]{hyperref}
\usepackage[fancyhdr,today,draft]{svninfo}
\pagestyle{fancyplain}\fancyhead[RE,LO]{\leftmark}\fancyhead[LE,RO]{\thepage}
\index{explicit indexing|usage}
\EnableCrossrefs
\PageIndex
\RecordChanges
\floatstyle{boxed}
\usepackage{filecontents}
\usepackage{dox}
\doxitem{PkgOption}{pkgoption}{package options}
\doxitem[idxtype=opt.]{CmdOption}{cmdoption}{command options}
\begin{document}
\svnInfo $Id$
\svnKeyword $HeadURL$
\DocInput{errata.dtx}
\end{document}
%%% CheckSum{508}
%</driver>
%<*linecounter.sty>
\NeedsTeXFormat{LaTeX2e}[1999/12/01]
\ProvidesPackage{linecounter}[]
\newread\File
\makeatletter
\newcommand{\CountLinesInFile}[2]{% Filename, counter name
  \@ifundefined{c@#2}%
  {\newcounter{#2}}{}
  \setcounter{#2}{-1}
  \openin\File=#1
  \@whilesw\unless\ifeof\File\fi{%
    \stepcounter{#2}
    \readline\File to \line
  }
  \closein\File
}
\makeatother
%</linecounter.sty>
%<*erratapgfkeysextra.tex>
%% Fix setting global values with pgfkeys
\long\def\pgfkeysgdef#1#2{%
  \long\def\pgfkeys@temp##1\pgfeov{#2}%
  \global\pgfkeyslet{#1/.@cmd}{\pgfkeys@temp}%
  \pgfkeyssetvalue{#1/.@body}{#2}%
}
\long\def\pgfkeysgaddvalue#1#2#3{%
  {%
    \toks0{#2}%
    \pgfkeysifdefined{#1}
    {\pgfkeys@temptoks\expandafter\expandafter\expandafter{\csname pgfk@#1\endcsname}}%
    {\pgfkeys@temptoks{}}%
    \toks1{#3}%
    \xdef\pgfkeys@global@temp{\the\toks0 \the\pgfkeys@temptoks \the\toks1}% believe or don't: the spaces are important
  }%
  \global\pgfkeyslet{#1}\pgfkeys@global@temp%
}
\pgfkeysdef{/handlers/.gcode}{\pgfkeysgdef{\pgfkeyscurrentpath}{#1}}
\pgfkeys{/handlers/.gappend/.code={\pgfkeysgaddvalue{\pgfkeyscurrentpath}{}{#1}}}
\pgfkeys{/handlers/.gstore in/.code={\pgfkeysalso{\pgfkeyscurrentpath/.gcode=\gdef#1{##1}}}}
\pgfkeys{/handlers/.gestore in/.code={\pgfkeysalso{\pgfkeyscurrentpath/.gcode=\global\edef#1{##1}}}}
%</erratapgfkeysextra.tex>
%<*strkeyformatter.sty>
\NeedsTeXFormat{LaTeX2e}[1999/12/01]
\ProvidesPackage{strkeyformatter}[2015/04/06 v1.0 Format a string based on key/value input]
\RequirePackage{xparse}
\RequirePackage{xstring}
\RequirePackage{stringstrings}
\RequirePackage{linecounter}
\RequirePackage{kvoptions}[2009/07/21]
\SetupKeyvalOptions {
  family = StrFmt,
  prefix = @StrFmtOpt@,
}
\DeclareStringOption[-]{notemptyprefix}
\DeclareStringOption[+]{evenemptyprefix}
\DeclareStringOption[@StrFmt@]{family}
\ProcessKeyvalOptions{StrFmt}
%%
\pgfkeysdefnargs{/handlers/.setStrKeyFmt}{2}{%
  \pgfkeysdefargs{\pgfkeyscurrentpath/.@formatter}{#1}{%
    \pgfkeyssetvalue{\pgfkeysdefaultpath.@formatted/.@def}{#2}}}
%%
\pgfkeysdefnargs{/handlers/.@StrFmt@setdefaultformat}{4}{%
  \xintFor ##9 in {#4} \do {%
    \pgfkeys{{#1/##9/.setStrKeyFmt={#2}{{#3}}}}% The default formater
  }}
\pgfkeys{/handlers/.also/.code={\pgfkeysalso{#1}}}
\pgfkeys{/handlers/.ealso/.code={\pgfqkeysalso{\pgfkeyscurrentpath}{.also/.expand once={#1}}}}
%%
\def\@StrFmt@set{\pgfqkeys{/\@StrFmtOpt@family}}
\DeclareExpandableDocumentCommand{\@StrFmt@setvalues}{ O{} m } {\@StrFmt@set{#1/.cd, #2}}
\DeclareExpandableDocumentCommand{\@StrFmt@getvalue}{ m } {\pgfkeysvalueof{\@StrFmtOpt@family/#1}}
\NewDocumentCommand{\SetDefaultKeylistFormat}{ O{#1} m m }{% format, group, keys
  \@StrFmt@set{#2/.@StrFmt@setdefaultformat={\@StrFmtOpt@family/#2}{#1}{#1}{#3}}%
}
\NewDocumentCommand{\StrFmtgetformattedee}{ m m m }{% key, args, macroname
  \@StrFmt@setvalues[#1]{.@formatter/.expand twice={#2}}%
  \pgfkeysgetvalue{/\@StrFmtOpt@family/#1/.@formatted/.@def}{#3}%
}
\NewDocumentCommand{\StrFmtgetformattede}{ m m m }{% key, args, macroname
  \@StrFmt@setvalues[#1]{.@formatter/.expand once={#2}}%
  \pgfkeysgetvalue{/\@StrFmtOpt@family/#1/.@formatted/.@def}{#3}%
}
\NewDocumentCommand{\StrFmtgetformatted}{ m m m }{% key, args, macroname
  \@StrFmt@setvalues[#1]{.@formatter={#2}}%
  \pgfkeysgetvalue{/\@StrFmtOpt@family/#1/.@formatted/.@def}{#3}%
}
\NewDocumentCommand{\StrFmtSetKeyFormat}{ O{} m m }{% PGF argument pattern, key, formated output
  \@StrFmt@set{#2/.setStrKeyFmt={#1}{#3}}% Set custom format
}
\NewDocumentCommand{\@StrFmt@isnotempty}{ m m }{\ifthenelse{\equal{#1}{}}{}{#2}}
\NewDocumentCommand{\@StrFmt@isempty}{ m m }{\ifthenelse{\equal{#1}{}}{#2}{}}
%%
\newcommand\@StrFmt@ifiskey[1]{\ifboolexpr{ test{\IfBeginWith{#1}{\@StrFmtOpt@evenemptyprefix}} or test{\IfBeginWith{#1}{\@StrFmtOpt@notemptyprefix}}}}
%%
\@StrFmt@set{formateroutputt/.initial={}}
%% Arguments:
%% [1]: [optional]macro name to store results
%% [2]: The command parent key for the keys
%% [3]: format group key
%% [4]: Keywords (as comma separated values)
%% [5]: Text containing the keys that define this formatted text to produce
\NewDocumentCommand{\@StrFmt@FormatKeywords}{ O{} m m m m }{% format group key, parent key, formatkey parent , keys/values, format-string
\@StrFmt@set{cmdkeys/.ealso=#4}%
\@StrFmt@set{formateroutputt={}}%
%% Reset arg key values
\xintFor ##1 in {#5} \do {%
  \ifboolexpr{ test {\@StrFmt@ifiskey{##1}} }%
  {\StrGobbleLeft{##1}{1}[\@StrFmt@tmp@right]%
    \pgfkeysifdefined{/\@StrFmtOpt@family/#3/\@StrFmt@tmp@right}{\@StrFmt@set{#3/\@StrFmt@tmp@right={}}}{}%
  }{}% End IfBeginWith
}%
%%
\@StrFmt@isnotempty{#4}{%
  %% If key/val options, set the values
  \xintFor ##1 in {#4} \do {%
    \@StrFmt@isnotempty{##1}{\@StrFmt@set{#3/##1}}%
  }}%
%% Process format keys
\xintFor ##9 in {#5} \do {%
  \@StrFmt@isnotempty{##9}{% Just for safety
    \ifboolexpr{ test {\@StrFmt@ifiskey{##9}} } {% Test if it's a key
      \StrGobbleLeft{##9}{1}[\@StrFmt@tmp@right]%
      \pgfkeysgetvalue{/\@StrFmtOpt@family/#3/\@StrFmt@tmp@right}{\@StrFmt@tmpvalue}%
      %% Fetch the value for the key
      %% If the key is prefixed with \@StrFmtOpt@notemptyprefix, and the value is empty, it's ignored
      \ifboolexpr{ not test{\IfBeginWith{#1}{\@StrFmtOpt@notemptyprefix}}
        and not test{\ifthenelse{\equal{\@StrFmt@tmpvalue}{}}}}{%
        \StrFmtgetformattede{#2/\@StrFmt@tmp@right}{\@StrFmt@tmpvalue}{\@StrFmt@tmp@fmtvalue}%
        \@StrFmt@set{formateroutputt/.append/.expand once={\@StrFmt@tmp@fmtvalue}}%
      }{}%
    }{% Not a key
      \@StrFmt@set{formateroutputt/.append={##9}}%
}}}%
\@StrFmt@set{formateroutputt/.get=\@StrFmt@formateroutputt}%
\ifthenelse{\equal{#1}{}}{%
%% Output the result here in the document
\@StrFmt@formateroutputt%
}{%
%% Store the in macro from ARG1
\protected@csxdef{#1}{\expandonce{\@StrFmt@formateroutputt}}%
}}%
%</strkeyformatter.sty>
%<*erratadocutil.tex>
\usepackage[fleqn,tbtags]{mathtools}
\usepackage{xparse}
\usepackage{caption}
\usepackage{float}
\PassOptionsToPackage{outputdir=build}{minted}
\newfloat{exfig}{thp}{lop}
\floatname{exfig}{Example}
%
\tcbset{latexcode/.style={
    listing only,
    fontlower=\small,
    skin=beamer,
    lower separated=true,
    enlarge top by=0.5cm,          % Add space above the box
    arc=0cm,            % Remove the rounded corners
    outer arc=0.3mm,    % Remove the rounded corners on outside
    right=0mm,% Remove spacing on right s
    top=5pt,
    bottom=5pt,
    boxrule=0mm,        % The border width
    bottomrule=1pt,
    leftrule=0.7pt,
    rightrule=1pt,
    lines before break=6, % Require at least 5 lines in the upper part to break the code block
    toptitle=2pt,       % Add space in title bar
    bottomtitle=1pt,    % Add space in title bar
    title={Example},%
    listing engine=minted,
    minted language=latex,
    before skip=\medskipamount,
    after skip=\medskipamount,
  },
  erratadocexample/.style={
    fontlower=\normalsize,
    minted options={fontsize=\normalsize\ttfamily},
  },
  codeandresult/.style={
    listing and text,
    erratadocexample,
    left=5pt,
    listing engine=minted,
  },
}
%
\DeclareTCBInputListing[%
auto counter,
list inside=exfig,
Crefname={Example}{Examples},
crefname={example}{examples},
]{\examplecodeinput}{ m m }{listing file={build/#1}, #2}
%
\DeclareTCBListing[%
auto counter,
list inside=exfig,
Crefname={Example}{Examples},
crefname={example}{examples},
]{examplecode}{ m }{#1}
\usepackage{etextools}% Adds the \expandnext macro
\usepackage{listings}
\makeatletter%
\lst@RequireAspects{writefile}
%
%
\lstnewenvironment{savecodeexample}[2][]{%
  \lstset{#1}%
  %\immediate\typeout{HELLO: savecodeexample}%
  % The content is stored in a temp box to avoid any spaces etc.
  % to have an impact
  \setbox\@tempboxa\hbox\bgroup
  % Write file to given filename
  \lst@BeginWriteFile{build/#2}%
}
{%
  \lst@EndWriteFile% closes output file
  \egroup% End of temp box
  % You can already read the file again here
%%  \input{build/#2}%
}
%
%% Simply do not include if the file does not exist. This is necessary
%% as the code listings from the source code aren't yet genereted on the
%% first run.
\NewDocumentCommand{\inputfromcode}{ m }{%
\IfFileExists{#1}{%
\CountLinesInFile{#1}{FileLines}%
\addtocounter{FileLines}{-1}%
\protected@edef\LastLine{\theFileLines}%
\expandnext{\custominputlisting{#1}{2}}{\LastLine}%
}{}%
}
\NewDocumentCommand{\custominputlisting}{ m m m }{%
  \tcbinputlisting{listing file={build/#1},
    listing only,
    minted options={
      fontsize=\footnotesize\ttfamily,
      firstline=#2,
      lastline=#3,
    },%
  }
}
\makeatother%
%%
% Redefine thempfootnote to fix footnote numbering inside tcolorboxes
\renewcommand{\thempfootnote}{\arabic{mpfootnote}}
\NewDocumentCommand{\docaption}{ O{exfig} m }{{\captionsetup{type=#1,position=below,belowskip=12pt} \caption{#2}}}
%</erratadocutil.tex>
%<*preamble.tex>
\usepackage{bookmark}
\PassOptionsToPackage{log-declarations=false}{xparse}
\usepackage{xparse}
\usepackage{cleveref}
\usepackage{tcolorbox}[2015/01/14] % Minimum v3.40
\tcbuselibrary{most}
\tcbuselibrary{documentation}
\tcbuselibrary{breakable}
\tcbuselibrary{skins}
\usetikzlibrary{shadings}
\tcbuselibrary{fitting}
\tcbuselibrary{hooks}
\tcbuselibrary{xparse}
\tcbuselibrary{minted}
\usetikzlibrary{shadings}
\tcbset{index format=doc,
  index colorize=true,
index annotate=true,
}
%%
\makeatletter%
\tcbset{
  color code/.store in=\kvtcb@col@code,
  color code=Definition,
  doc head code/.style={doc@head@code/.style={#1}},
  %doc head/.style append={doc head code={#1}},
  doc head/.style={doc head command={#1},doc head environment={#1},doc head key={#1},
    doc head code={#1}},
}
\def\tcb@Print@Code#1{\textcolor{\kvtcb@col@code}{\bfseries{#1}}}

\NewTColorBox[auto counter, crefname={definition}{definitions}]{docTargetBox}{ O{} m } {
title={\thesection.\thetcbcounter~#2}, #1}

\newenvironment{docCode}[3][]{\tcbset{#1}%
  \begin{tcb@manual@entry}%
  \begin{tcb@doc@head}{doc@head@command}
  \tcb@Print@Code{#2}\label{code:#3}%
  \tcb@doc@do@description%
  \end{tcb@doc@head}\tcbset{before upper=}}%
  {\end{tcb@manual@entry}}

\newenvironment{docCode*}{\bgroup\tcb@doc@toindexfalse\begin{docCode}}{\end{docCode}\egroup}

\DeclareDocumentCommand\tcb@ref@docc{msm}{%
  \hyperref[#1:#3]{\texttt{#3}}%
  \IfBooleanTF{#2}{}{%
    \ifnum\getpagerefnumber{#1:#3}=\thepage%
    \else%
      \textsuperscript{\ding{213}\,\kvtcb@text@pageshort\,\pageref*{#1:#3}}%
    \fi}}%

\def\refCode{\tcb@ref@docc{code}}
\makeatother%

\DeclareDocumentEnvironment{docTarget}{ m } % Label name
{\hypertarget{#1}{}
\bookmark[dest={#1},level=2]{#1}
\begin{docCode*}[color key=yellow!50!red,title=TEST]{#1}{#1}
}
{\end{docCode*}}
%</preamble.tex>
% \fi
%
% \changes{v0.1}{2006/07/18}{First Version with Documentation}
% \changes{v0.2}{2006/08/01}{More Documentation and Keywords Management}
% \changes{v0.3}{2006/11/21}{adding type Keyword}
% \changes{v0.4}{2015/04/13}{moved to GIT, fixing switches}
% \changes{v1.0}{2015/04/16}{Added support for links with hyperref, and customization through formatters}
%
% \GetFileInfo{errata.sty}
%
% \MakeShortVerb{\|}
% \title{Recording Errata in {\LaTeX} Documents\thanks{Version {\fileversion} (last revised
%        {\filedate})}}
%    \author{Michael Kohlhase\\
%            Computer Science, Jacobs University,  Bremen, Germany\\
%            \url{http://kwarc.info/kohlhase}\\[6mm]Bendik Rønning Opstad\\(bro.devel@gmail.com)}
% \maketitle
%
% \begin{abstract}
%   This package provides a simple infrastructure for recording errata in {\LaTeX}
%   documents. This allows to maintain an updated version of the document (with all errors
%   corrected) and automatically generate an errata document highlighting the difference
%   to the published version.
% \end{abstract}
%
% \section{Introduction}\label{sec:intro}
%
% The life-cycle of a document does not end with its publication. After that, errors will
% be discovered, and have to be managed. The best way to do this is by marking errata in
% the text and generating the errata document from that.
%
% \section{Package Options}\label{sec:pkgopt}
%
% The |errata| package can be called with a variety of package options.
%
% \begin{docKey}{show}{=\meta{true/false}}{boolean, initially \colOpt{false}}
%   Governs the general visibility of the errata markup. This option enables the options
%   \refKey{foots}, \refKey{margins}, \refKey{inlinechanges}, and \refKey{inlinemarks}.
% \end{docKey}
%
% \begin{docKey}{hide}{=\meta{true/false}}{complementary option to \refKey{show}}
%   Will do the opposite of \refKey{show}, ensuring that the errata markup is totally
%   invisible in the formatted document.
% \end{docKey}
%
% \begin{docKey}{margins}{=\meta{true/false}}{boolean, initially \colOpt{false}}
%   Enables decorations in the margin for each erratum. Setting
%   \colDef{margins}=\colOpt{false} will globally disable all margin notes.
% \end{docKey}
%
% \begin{docKey}{foots}{=\meta{true/false}}{boolean, initially \colOpt{false}}
%   Enables footnote decorations for each erratum.
% \end{docKey}
%
% \begin{docKey}{inlinemarks}{=\meta{true/false}}{boolean, initially \colOpt{true}}
%   By default, each errata change is marked in the document with the correction
%   encapsulated by paranthesis as shown in \cref{code:short-errata}. By setting
%   \colDef{inlinemarks}=\colOpt{false}, the inline marks will not be added.
% \end{docKey}
%
% \begin{docKey}{inlinechanges}{=\meta{true/false}}{boolean, initially \colOpt{true}}
%   The default behavior is to make the errata changes visible in the document inlined in
%   the text. By setting \colDef{inlinechanges}=\colOpt{false}, the document text will
%   stay unchanged.
% \end{docKey}
%
% \begin{docKey}{record}{=\meta{true/false}}{boolean, default \colOpt{true}}
%   The errata entries will written to a file named \meta{jobname}|-errata.tex| so that
%   they can be included in an errata document (see \cref{sec:generating}).
% \end{docKey}
%
% \begin{docKey}{hyperref}{=\meta{true/false}}{boolean, initially \colOpt{true}}
%   This option is used to control whether the \refAux{hyperref} package will be used to
%   create hyperlinks between each erratum, the margin notes and the errata list.
% \end{docKey}
%
% \begin{docKey}{cleveref}{=\meta{true/false}}{boolean, initially \colOpt{true}}
%   This package option will require the \refAux{cleveref} package, and changes how labels
%   are created so that the registered errata corrections can be referenced with
%   \refAux{cleveref} commands\footnote{See \refCode{Cleveref support} for the code
%   definition.}.
% \end{docKey}
%
% \begin{docKey}{marginnote}{=\meta{true/false}}{boolean, initially \colOpt{true}}
%   The default behavior is to use the \refAuxcs{marginpar} macro to create the errata
%   reference in the margins. In some environments, such as floating environments, the
%   \refAuxcs{marginpar} macro cannot not be used. By default, \refAuxcs{marginnote} from
%   the \refAux{marginnote} package will be used to handle margin notes when necessary.
% \end{docKey}
%
% \begin{docKey}{styles}{=\meta{style name}}{string, initially default}
%   The default styling of the errata markup in the errata list and the footnotes is very
%   simple. This package option can be used to specify one or more predefined styles.
%   Available styles are:
%   \colOpt{default}, \colOpt{detailed}, \colOpt{tablelist}.
% \end{docKey}
%
%
% \begin{docKey}{requirecmddescription}{=\meta{true/false}}{boolean, initially \colOpt{true}}
%   By setting this package option to \colOpt{false}, the errata command macros
%   (\refCom{erratumAdd}, \refCom{erratumDelete}, \refCom{erratumReplace} and
%   \refCom{erratumNote}) are defined without the mandatory description argument, i.e.
%   \refCom*{erratumAdd}\oarg{keys}\marg{add}
% \end{docKey}
%
% \newpage
% \subsection{Default command arguments}\label{sect:defargs}
%
% The following package options can be used to specify a set of default settings to be
% passed to the errata commands described in \cref{sec:metadata}. The different settings
% are described in \cref{sec:cmdopt}.
%
% The default arguments may be changed using the following package options:
%
% \begin{docKey}{defaultargs}{=\brackets{\meta{command options}}}{key/values, default
%   described below}
%   This option allows setting a set of the default arguments to be passed to the errata
%   commands \refCom{erratumAdd}, \refCom{erratumReplace}, \refCom{erratumDelete}, and
%   \refCom{erratumNote}.\\
%
%
%   The default values passed to all the commands are as follows\footnote{The code
%   defining the default settings are defined in \refCode{Default command arguments}.}:
%
%   \colDef{margin}=\colOpt{true}\\
%   \colDef{foot}=\colOpt{true}\\
%   \colDef{inlinemark}=\colOpt{true}\\
%   \colDef{inlinechange}=\colOpt{true}\\
%   \colDef{record}=\colOpt{true}\\
%   \colDef{marginargs}=\colOpt{\brackets{}}\\
% \end{docKey}
%
% \begin{docKey}{defaultargsadd}{=\brackets{\meta{command options}}}{key/values, initially empty}
% This option allows setting a set of the default arguments to be passed to the
% \refCom{erratumAdd} command. Any arguments set will override the respective
% argument set with \refKey{defaultargs}.
% \end{docKey}
%
% \begin{docKey}{defaultargsdelete}{=\brackets{\meta{command options}}}{key/values, initially empty}
% This option allows setting a set of the default arguments to be passed to the
% \refCom{erratumDelete} command. Any arguments set will override the respective
% argument set with \refKey{defaultargs}.
% \end{docKey}
%
% \begin{docKey}{defaultargsreplace}{=\brackets{\meta{command options}}}{key/values, initially empty}
% This option allows setting a set of the default arguments to be passed to the
% \refCom{erratumReplace} command. Any arguments set will override the respective
% argument set with \refKey{defaultargs}.
% \end{docKey}
%
% \begin{docKey}{defaultargsnote}{=\brackets{\meta{command options}}}{key/values, initially empty}
% This option allows setting a set of the default arguments to be passed to the
% \refCom{erratumNote} command. Any arguments set will override the respective
% argument set with \refKey{defaultargs}.
% \end{docKey}
%
%
% \section{Errata commands}
%
% Marking errata changes in the {\LaTeX} text is done using a few simple macros.
%
% \begin{docCommand}{erratumAdd}{\oarg{keys}\marg{desc}\marg{add}}
% marks up a correction by adding the text \meta{add} for addition with
% \meta{desc} describing the intended correction.
% \end{docCommand}
%
% \begin{docCommand}{erratumDelete}{\oarg{keys}\marg{desc}\marg{delete}}
% marks up a correction by deleting the text \meta{delete} with \meta{desc}
% describing the intended correction.
% \end{docCommand}
%
% \begin{docCommand}{erratumReplace}{\oarg{keys}\marg{desc}\marg{old}\marg{new}}
% marks up a correction by replacing the text \meta{old} with \meta{new}.
% \end{docCommand}
%
% \begin{docCommand}{erratumNote}{\oarg{keys}\marg{desc}\marg{note}}
% creates a note entry that cannot be described by the other commands.
% \end{docCommand}
%
% \subsection{Errata example}
%
% \Cref{code:short-errata} shows all editing operations in action.
%
% \begin{savecodeexample}[gobble=2]{example2.tex}
%  Here we have \erratumAdd{forgotten word}{three} errata in one \erratumDelete{
%  superfluous phrase}{darned} long \erratumReplace{translated}{Zeile}{line}.
% \end{savecodeexample}
%
% \examplecodeinput{example2.tex}{latexcode, label={code:short-errata}}
% \docaption{Some short errata}
%
%
% All of these macros mark the location of the errata in the margin and document the
% changes in footnote-like structures. The text in \cref{code:short-errata} is rendered as
% follows:
% \begin{quote}
%   Here we have \erratumAdd{forgotten word}{three} errata in one
%   \erratumDelete{superfluous phrase}{darned} long
%   \erratumReplace{translated}{Zeile}{line}
% \end{quote}
%
% If the package option \refKey{hyperref} is enabled, the margin note will link to the
% entry in the errata list produced by calling \refCom{PrintErrata}.
%
% \subsection{Errata environment}
%
% \begin{docEnvironment}{erratum}{\oarg{keys}{\marg{desc}}}
%   Extended errata group multiple editing operations into a coherent group via the
%   \refEnv{erratum} environment. The first argument of this environment is an explanation
%   as for the short errata above. The \refEnv{erratum} environment provides local
%   versions of the editing markup macros, which behave like those, but lack the first
%   (explanation) argument, which is already given in the environment that contains them.
% \end{docEnvironment}
%
% \begin{savecodeexample}[gobble=4]{example-erratum-env.tex}
%   \begin{erratum}[date=2006-07-19,reported-by=Michael Kohlhase]{old should be new}
%   This is a test of a long erratum
%   \begin{enumerate}
%     \item We can replace \eReplace{oldtext}{newtext}
%     \item and \eAdd{new text}
%     \item and finally delete old text\eDelete{alltogether}
%     \end{enumerate}
%   \end{erratum}
% \end{savecodeexample}
%
% \examplecodeinput{example-erratum-env.tex}{latexcode, label={code:extended-erratum}}
% \docaption{An extended erratum with local correction markers}
%
% The text in \cref{code:extended-erratum} would be rendered as
% \begin{quote}
% \begin{erratum}[reported-by=Michael Kohlhase,date=2006-11-21]{old should be new}
% this is a test of a long erratum
% \begin{enumerate}
%   \item We can replace \eReplace{oldtext}{newtext}
%   \item and \eAdd{new text}
%   \item and finally delete old text\eDelete{alltogether}
%   \end{enumerate}
% \end{erratum}
% \end{quote}
%
% The |erratum| environment should also be used in situations where the error occurs in an
% environment, where normal {\TeX/\LaTeX} processing is suspended, e.g. a |verbatim|
% environment. In this case, we can use it to attach correction information via the
% environment, but do not use the local change documentations.
%
% \section{Command Options}\label{sec:cmdopt}
%
% Each of the errata commands take an optional list of comma separated key/values as argument.
% This can be used to specify per-command options for each errata entry.
%
% \begin{docKey}{margin}{=\meta{true/false}}{boolean, defaults to \colOpt{true}}
%   Allows to override the default \colDef{margin} value for this erratum.
% \end{docKey}
%
% \begin{docKey}{foot}{=\meta{true/false}}{boolean, defaults to \colOpt{true}}
%   Allows to override the default \colDef{foot} value for this erratum.
% \end{docKey}
%
% \begin{docKey}{inlinechange}{=\meta{true/false}}{boolean, defaults to \colOpt{true}}
%   Allows to override the default \colDef{inlinechange} value for this erratum.
% \end{docKey}
%
% \begin{docKey}{inlinemark}{=\meta{true/false}}{boolean, defaults to \colOpt{true}}
%   Allows to override the default \colDef{inlinemark} value for this erratum.
% \end{docKey}
%
% \begin{docKey}{label}{=\meta{name of label}}{initially empty}
%   Allows specifying a custom label name to be registered for this errata entry.
% \end{docKey}
%
% \begin{docKey}{uselabel}{=\meta{name of label}}{string, defaults to errata<X>}
%   Allows specifying a label to use instead of the default label generated for each
%   registered erratum entry (errata<X> where X is the erratum number). This is necessary
%   in environments where only one label may exist, such as in the \refAux{equation}
%   environment.
% \end{docKey}
%
% \begin{docKey}{marginargs}{=\brackets{\meta{key/valoptions}}}{list, initially empty}
%   Allows specifying custom arguments to be passed to the marginnote formatter.
% \end{docKey}
%
% \begin{docKey}{usemarginnote}{[=\meta{length}]}{length, initially \colOpt{0pt}}
%   Forces the use of \refAuxcs{marginnote} for the errata command. An optional value can
%   be specified which denotes how much to verticaly shift the margin note.
% \end{docKey}
%
%
% \section{Errata metadata}\label{sec:metadata}
%
% \subsection{Data keys}\label{subsec:data-keys}
%
% For each errata command, a set of meta-data can be specified for an errata entry.
% The following keys can be used to registered additional data:
%
% \begin{docKey}{date}{=\meta{date when registered}}{string, initially empty}
%   Allows specifying the date when the erratum was registered.
% \end{docKey}
%
% \begin{docKey}{reported-by}{=\meta{name of reporter}}{string, initially empty}
%   Allows specifying the name of the reporter.
% \end{docKey}
%
% \begin{docKey}{description}{=\meta{description}}{string, initially empty}
%   Allows specifying the description of the change.
%
%   If the package option \refKey{requirecmddescription} is true (the default), This key
%   defaults to the value specified in the \meta{desc} field.
% \end{docKey}
%
% \begin{docKey}{location}{=\meta{location hint}}{string, initially empty}
%   Used to specify the location in the text where the change is made.
% \end{docKey}
%
% \begin{docKey}{type}{=\meta{type of correction}}{string, initially empty}
%   Used to specify what type of change is registered, e.g. grammar, clarification,
%   layout.
% \end{docKey}
%
% Remember to encapsulate any of values in curly braces if it contains a comma.
%
%
% \subsection{Formatter keys}\label{subsec:formatter-keys}
%
% For each of the metadata keys specified in \cref{subsec:data-keys}, a formatter
% is defined for the errata list and the footnotes. The formatter defines a custom
% formatting rule which will be applied to the input value before being used.
%
% Additionally, the following formatter keys exist which may \textbf{not} be passed as
% argument to the errata commands:
%
% \begin{docKey}{pageref}{}{name of label, errata<X>}
%   Formats a reference to the page number where the erratum is located.
% \end{docKey}
%
% \begin{docKey}{actiondescription}{}{}
%   the formatted description of the action
% \end{docKey}
%
% \begin{docKey}{action}{}{}
%   The action name
% \end{docKey}
%
% \begin{docKey}{erratumlistitem}{}{}
%   This key is used to insert an \refAuxcs{item[]} in the errata list.
% \end{docKey}
%
% \begin{docKey}{erratumlistitemtitle}{}{}
%   The title for the erratum in the errata list
% \end{docKey}
%
% \begin{docKey}{erratumlisttarget}{}{}
%   The name used when creating a \refAuxcs{label} target in the errata list.
% \end{docKey}
%
% \begin{docKey}{marginfmt/marginnote}{}{}
%   Formats the marginnote
% \end{docKey}
%
%
% \subsection{Setting the format for footnotes and errata list}\label{subsec:formatting-foot-and-llist}
%
% The formatting of the footnotes and errata list can be changed using the two macros
% \refCom{errataSetListFormat} and \refCom{errataSetFootFormat}.
%
% The argument \meta{keys} should specify a comma separated list of keys as specified in
% \cref{sec:metadata}\footnote{See \refCode{Errata default style definition} for
% the code definition}.
%
% \begin{docCommand}{errataSetFootFormat}{\marg{keys}}
%   Allows changing how the footnotes are formatted.
%
% ^^A Include code extracted from the source code
% {\inputfromcode{default-footnote-format.tex}}
%
% \end{docCommand}
%
% \begin{docCommand}{errataSetListFormat}{\marg{keys}}
%
% The default format for the errata list is defined with:
%
% ^^A Include code extracted from the source code
% {\inputfromcode{default-erratalist-format.tex}}
%
% \end{docCommand}
%
%
% \subsection{Formatting the keys}\label{subsec:formatting-keys}
%
% The format of each of the keys can be set using the macro
% \refCom{errataSetKeyFormat}.
%
% \begin{docCommand}{errataSetKeyFormat}{\marg{key}\marg{format}}
%   should be used to set the formatter key for a meta data key.
% \end{docCommand}
%
% \iffalse
% Must use a group around input to avoid the font being changed..????
% \fi
% The default format for the keys are as as follows\footnote{This code is defined at \refCode{Default formatters}.}:
% ^^A Include code extracted from the source code
% {\inputfromcode{style-default-formats.tex}}
%
%
% \section{Generating Statistics and Errata Documents}\label{sec:generating}
%
% \begin{docCommand}{PrintErrata}{}
%   This command inserts the errata list at the current position.
% \end{docCommand}
%
% \begin{docCommand}{printerrata}{\oarg{postfix(-errata.tex)}\marg{prefix}}
%   This command allows to print the errata for another document. This command is useful
%   when generating errata documents for published works. Say we have a book with a driver
%   file |thebook.tex|, into which we have incorporated errata markup using the
%   infrastructure detailed above. Then we have a new document called e.g. |theerrata.tex|
%   which has the form given in \cref{code:sample-errata-document}. Note that we have used
%   \refCom{printerrata}\brackets{thebook} to include the errata notices generated from
%   |thebook.tex|.
% \end{docCommand}
%
%
% \begin{docEnvironment}{errata}{}
%   The errata list produced when the \refKey{record} option is enabled is placed inside
%   this environment.
% \end{docEnvironment}
%
% \begin{docCommand}{erratumItem}{\oarg{macro}\marg{title}\marg{text}}
%   Errata can be marked up using the \refCom{erratumItem} macro in the \refEnv{errata}
%   environment. \refCom{erratumItem} takes one optional, and two required arguments.
%
%   The first argument (\oarg{macro}) can be used to specify a macro containing a value
%   that will be expanded after the other values, i.e. at the end.
%
%   The second argument \marg{title} will be formatted using the listformats/
%   \refKey{erratumlistitem} key formatter. For the default formatting of the errata list
%   this is used to insert a |\item[|\meta{title}|]| before the remaining errata meta data
%   specified in the third argument \marg{text}
%
%   \begin{tcolorbox}[marker]
%     Normally it is not necessary to manually use this macro, as the \refKey{record}
%     option will write an \refAux{erratumItem} the to the errata for each registered
%     erratum.
%   \end{tcolorbox}
% \end{docCommand}
%
% \begin{docCommand}{erratamessage}{}
%   Putting the macro \refCom{erratamessage} just before the |\end{document}| will
% generate a message with cardinality information for the errata into the log file.
% \end{docCommand}
%
% In the errata document in \cref{code:sample-errata-document} we postulate that we
% keep an updated version of The Book online\footnote{And indeed it is good practice to do
%   so if the copyright agreement with the publisher allows this.} using the
% infrastructure provide by the |errata| package. In the
% updated version of |thebook.tex|, it can be useful to tabulate the errata as well,
% e.g. in a section in the appendix. This can be done by the \refCom{PrintErrata} command. Note
% that this command needs to close the errata file |thebook-errata.tex| therefore we need
% a |\newpage| to clear the queue of waiting |\writes| before |thebook-errata.tex| can be
% loaded (otherwise we may be missing the errata from the last page).
%
% \begin{savecodeexample}[gobble=4]{example1.tex}
%   \documentclass{article}
%   \usepackage[hide]{errata}
%   \title{Errata for The Book}
%   \begin{document}
%   \maketitle
%   \begin{abstract} This document tracks the errata in The Book. \end{abstract}
%   \section{Introduction}
%   The errata of The Book are tracked in this document, whose newest version
%   can be found at \url{.../berrata.pdf}. A version of The Book that contains
%   all errata corrections (and markup of what changed) can be found at
%   \url{.../book.pdf}.
%
%   In the following we will tabulate the errata in document order. Their
%   location will be referenced by the section they appear in rather than the
%   page number, since we do not expect the former to change in the errata
%   correction process.
%
%   \section{The Errata in The Book}
%   \printerrata{thebook}
%   \end{document}
% \end{savecodeexample}
%
% \examplecodeinput{example1.tex}{latexcode, label={code:sample-errata-document}}
% \docaption{A Sample Errata Document}
%
%
% \section{Limitations}\label{sec:limits}
%
% If you want to contribute, please contact the author or make a pull request at
% \url{https://github.com/KWARC/LaTeX-errata}
%
% \subsection{Support for special environments}\label{subsec:special-environments}
%
% When registering an erratum in environments such as \refAux{equation}
% (from \refAux{amsmath} package),
% labels cannot be created freely, therefore the label already registered in the
% environment must be passed to the errata command.
%
% \Cref{code:example-equation} shows example where the label is specified manually.
%
% \begin{savecodeexample}[gobble=4]{example-equation.tex}
%   \begin{equation}\label{customlabel}
%   \begin{split}
%     a=a  \\
%     b=b \erratumAdd[uselabel=customlabel]{label}{custom label example}
%   \end{split}
%   \end{equation}
% \end{savecodeexample}
%
% \examplecodeinput{example-equation.tex}{codeandresult,
% label={code:example-equation}, title=Title}
% \docaption{Some short errata}
%
% \subsection{Floating environments}\label{sec:limits-floats}
%
% When registering errata corrections inside floating environments, some manual
% adjustments may be required. \Cref{code:example-margin-adjustment} shows
% how the margins can be adjusted manually.
%
%
% \begin{savecodeexample}[gobble=4]{example-margin-adjustment.tex}
%   In this sentence\erratumAdd[inlinemark=false,label={comma1}]{added comma}
%   {,} we added a comma with inline mark
%   disabled\erratumAdd[inlinemark=true,label={comma2}] {added another
%   comma}{,} as well as a comma with inlinemarking.
%
%   Inside environments where \refAuxcs{marginnote} must be used, it may be
%   necesssary to manually adjust the vertical placement of the margin note
%   when multiple notes overlap.
%
%   As this code is processed inside a \refAux{tcolorbox}, \refAuxcs{marginnote}
%   must be used instead of \refAuxcs{marginpar}. As we can see, \Cref{comma1}
%   and \cref{comma2} are overlapping in the margin as
%   \refAuxcs{marginnote} does not handle stacking multiple notes. Therefore,
%   we must manually ask \refAuxcs{marginnote} to shift the second margin
%   value, such as with \erratumAdd[inlinemark=true] {added missing
%   word}{these} two \erratumAdd[inlinemark=true,usemarginnote=20pt] {added
%   another missing word}{corrections}.
% \end{savecodeexample}
%
% \examplecodeinput{example-margin-adjustment.tex}{codeandresult,
% label={code:example-margin-adjustment}, title=Title}
%   \docaption{Example showing how to adjust the margin notes}
%
%
% \StopEventually{%
% \newpage\PrintIndex
% \newpage \PrintChanges
% }
%
% \newpage
% \section{The Implementation}
%
%\iffalse
%    \begin{macrocode}
%<*package>
%    \end{macrocode}
%\fi
%    \begin{macrocode}
\NeedsTeXFormat{LaTeX2e}[1999/12/01]
\ProvidesPackage{errata}[2015/04/16 v1.0 Errata Management]
\RequirePackage{xifthen}
\RequirePackage{pgfkeys}
\RequirePackage{etoolbox}
\RequirePackage{xpatch}
\RequirePackage{xparse}
\RequirePackage{xinttools} % defines \xintFor macro
\RequirePackage[family=@errata@]{strkeyformatter}
\RequirePackage{kvoptions}[2009/07/21]
%    \end{macrocode}
% Setup package options
%    \begin{macrocode}
\SetupKeyvalOptions {
  family = Errata,
  prefix = @errataOpt@,
}
%    \end{macrocode}
%\begin{pkgoption}{show}
% The |show| package option shows everything
%    \begin{macrocode}
\DeclareBoolOption{show}
%    \end{macrocode}
%\end{pkgoption}
%\begin{pkgoption}{hide}
% The |hide| package option hides everything
%    \begin{macrocode}
\DeclareComplementaryOption{hide}{show}
%    \end{macrocode}
%\end{pkgoption}
%\begin{pkgoption}{margins}
% The |margins| package option
%    \begin{macrocode}
\DeclareBoolOption{margins}
%    \end{macrocode}
%\end{pkgoption}
%\begin{pkgoption}{foots}
% The |foots| package option
%    \begin{macrocode}
\DeclareBoolOption{foots}
%    \end{macrocode}
%\end{pkgoption}
%\begin{pkgoption}{record}
% The |record| package option shows everything
%    \begin{macrocode}
\DeclareBoolOption[true]{record}
%    \end{macrocode}
%\end{pkgoption}
% \changes{v1.0}{2015/03/30}{requirecmddescription package option added}
%\begin{pkgoption}{requirecmddescription}
% The |requirecmddescription| package option
%    \begin{macrocode}
\DeclareBoolOption[true]{requirecmddescription}
%    \end{macrocode}
%\end{pkgoption}
% \changes{v1.0}{2015/03/30}{hyperref package option added}
%\begin{pkgoption}{hyperref}
% The |hyperref| package option
%    \begin{macrocode}
\DeclareBoolOption[true]{hyperref}
%    \end{macrocode}
%\end{pkgoption}
% \changes{v1.0}{2015/04/12}{cleveref package option added}
%\begin{pkgoption}{cleveref}
% The |cleveref| package option
%    \begin{macrocode}
\DeclareBoolOption[true]{cleveref}
%    \end{macrocode}
%\end{pkgoption}
% \changes{v1.0}{2015/04/12}{marginnote package option added}
%\begin{pkgoption}{marginnote}
% The |marginnote| package option
%    \begin{macrocode}
\DeclareBoolOption[true]{marginnote}
%    \end{macrocode}
%\end{pkgoption}
% \changes{v1.0}{2015/04/12}{environmenttestmode package option added}
%\begin{pkgoption}{environmenttestmode}
% The |environmenttestmode| package option allows to set when |\marginnnote| is used
% instead of |\marginpar|. Errata will try to detect when |\marginnnote| should be used
% instead of |\marginpar|, and this option allows for finetuning how safe the test is.
% The default (|environmenttestmode=0|) will only use |\marginnnote| when it knows for
% certain that it cannot use |\marginpar|. This is in |math mode|, in |inner mode|, and
% in tcolorbox's |tcb@savebox| environment. There are certainly many more that should be
% added to this list. By setting |environmenttestmode=1|, |\marginnnote| will be used
% whenever |\@captype| is defined, which should include most floating environments such
% as table and figure. The strictest mode |environmenttestmode=2| will force
% |\marginnnote| to be used whenever the current environment is not 'document', which
% means any environment outside of the root document.
%    \begin{macrocode}
\DeclareStringOption[0]{environmenttestmode}
%    \end{macrocode}
%\end{pkgoption}
% \changes{v1.0}{2015/04/16}{uniquelistlabels package option added}
%\begin{pkgoption}{uniquelistlabels}
% The |uniquelistlabels| package option
%% Disabled by default (0), if option set, default level is 3.
%    \begin{macrocode}
\DeclareStringOption[0]{uniquelistlabels}[3]
%    \end{macrocode}
%\end{pkgoption}
% \changes{v1.0}{2015/04/16}{inlinemarks package option added}
%\begin{pkgoption}{inlinemarks}
% The |inlinemarks| package option
%    \begin{macrocode}
\DeclareBoolOption[true]{inlinemarks}
%    \end{macrocode}
%\end{pkgoption}
% \changes{v1.0}{2015/04/16}{inlinechanges package option added}
%\begin{pkgoption}{inlinechanges}
% The |inlinechanges| package option
%    \begin{macrocode}
\DeclareBoolOption[true]{inlinechanges}
%    \end{macrocode}
%\end{pkgoption}
% \changes{v1.0}{2015/04/16}{styles package option added}
%\begin{pkgoption}{styles}
% The |styles| package option
%    \begin{macrocode}
\DeclareStringOption[default]{styles}
%    \end{macrocode}
%\end{pkgoption}
% \changes{v1.0}{2015/04/16}{defaultargs package option added}
%\begin{pkgoption}{defaultargs}
% The |defaultargs| package option
%    \begin{macrocode}
\DeclareStringOption[]{defaultargs}
%    \end{macrocode}
%\end{pkgoption}
% \changes{v1.0}{2015/04/16}{defaultargsadd package option added}
%\begin{pkgoption}{defaultargsadd}
% The |defaultargsadd| package option
%    \begin{macrocode}
\DeclareStringOption[]{defaultargsadd}
%    \end{macrocode}
%\end{pkgoption}
% \changes{v1.0}{2015/04/16}{defaultargsdelete package option added}
%\begin{pkgoption}{defaultargsdelete}
% The |defaultargsdelete| package option
%    \begin{macrocode}
\DeclareStringOption[]{defaultargsdelete}
%    \end{macrocode}
%\end{pkgoption}
% \changes{v1.0}{2015/04/16}{defaultargsreplace package option added}
%\begin{pkgoption}{defaultargsreplace}
% The |defaultargsreplace| package option
%    \begin{macrocode}
\DeclareStringOption[]{defaultargsreplace}
%    \end{macrocode}
%\end{pkgoption}
% \changes{v1.0}{2015/04/16}{defaultargsdelete package option added}
%\begin{pkgoption}{defaultargsnote}
% The |defaultargsnote| package option
%    \begin{macrocode}
\DeclareStringOption[]{defaultargsnote}
%    \end{macrocode}
%\end{pkgoption}
%    \begin{macrocode}
\ProcessKeyvalOptions{Errata}
%% Utility macros
%%
%% This is file `erratapgfkeysextra.tex',
%% generated with the docstrip utility.
%%
%% The original source files were:
%%
%% errata.dtx  (with options: `erratapgfkeysextra.tex')
%% 
%% Fix setting global values with pgfkeys
\long\def\pgfkeysgdef#1#2{%
  \long\def\pgfkeys@temp##1\pgfeov{#2}%
  \global\pgfkeyslet{#1/.@cmd}{\pgfkeys@temp}%
  \pgfkeyssetvalue{#1/.@body}{#2}%
}
\long\def\pgfkeysgaddvalue#1#2#3{%
  {%
    \toks0{#2}%
    \pgfkeysifdefined{#1}
    {\pgfkeys@temptoks\expandafter\expandafter\expandafter{\csname pgfk@#1\endcsname}}%
    {\pgfkeys@temptoks{}}%
    \toks1{#3}%
    \xdef\pgfkeys@global@temp{\the\toks0 \the\pgfkeys@temptoks \the\toks1}% believe or don't: the spaces are important
  }%
  \global\pgfkeyslet{#1}\pgfkeys@global@temp%
}
\pgfkeysdef{/handlers/.gcode}{\pgfkeysgdef{\pgfkeyscurrentpath}{#1}}
\pgfkeys{/handlers/.gappend/.code={\pgfkeysgaddvalue{\pgfkeyscurrentpath}{}{#1}}}
\pgfkeys{/handlers/.gstore in/.code={\pgfkeysalso{\pgfkeyscurrentpath/.gcode=\gdef#1{##1}}}}
\pgfkeys{/handlers/.gestore in/.code={\pgfkeysalso{\pgfkeyscurrentpath/.gcode=\global\edef#1{##1}}}}
\endinput
%%
%% End of file `erratapgfkeysextra.tex'.

\def\errataset{\pgfqkeys{/@errata@}}
\DeclareExpandableDocumentCommand{\@errata@setvalue}{ m } {\pgfkeys{/@errata@/#1}}
\DeclareExpandableDocumentCommand{\@errata@setvalues}{ O{} m } {\pgfkeys{/@errata@/#1.cd, #2}}
\DeclareExpandableDocumentCommand{\@errata@getvalue}{ m } {\pgfkeysvalueof{/@errata@/#1}}
\NewDocumentCommand{\@errata@isnotempty}{ m m }{\ifthenelse{\equal{#1}{}}{}{#2}}
\NewDocumentCommand{\@errata@isempty}{ m m }{\ifthenelse{\equal{#1}{}}{#2}{}}
\NewDocumentCommand{\@errata@ifstrequal}{ m m m }{\ifthenelse{\equal{#1}{#2}}{#3}{}}
%%
%    \end{macrocode}
% \begin{docTarget}{Default command arguments}
% Define the custom default arguments for the errata command
% \end{docTarget}
%
%    \begin{macrocode}
\pgfkeys{
  /@errata@/cmdkeys/.is family, /@errata@/cmdkeys,
  defaultargs/.style={record=true, foot=true, margin=true,
    marginargs={}, inlinemark=true, inlinechange=true},
  defaultargsadd/.style={},
  defaultargsreplace/.style={},
  defaultargsdelete/.style={},
  defaultargsnote/.style={},
  /handlers/.errataapplystyles/.code={\xintFor ##9 in {#1} \do {\errataset{styles/##9}}},
}
%% Add any default arguments specified by the user
\@errata@setvalues[cmdkeys/]{defaultargs/.append style/.expand once={{\@errataOpt@defaultargs}}}
\@errata@setvalues[cmdkeys/]{defaultargsadd/.append style/.expand once={\@errataOpt@defaultargsadd}}
\@errata@setvalues[cmdkeys/]{defaultargsreplace/.append style/.expand once={\@errataOpt@defaultargsreplace}}
\@errata@setvalues[cmdkeys/]{defaultargsdelete/.append style/.expand once={\@errataOpt@defaultargsdelete}}
\@errata@setvalues[cmdkeys/]{defaultargsnote/.append style/.expand once={\@errataOpt@defaultargsnote}}
%%
\if@errataOpt@show
\@errataOpt@marginstrue%
\@errataOpt@footstrue%
\@errataOpt@inlinemarkstrue%
\@errataOpt@inlinechangestrue%
\fi
%%
\@ifpackageloaded{hyperref}{}{\@errataOpt@hyperreffalse }
\@ifpackageloaded{cleveref}{}{\@errataOpt@clevereffalse }
\if@errataOpt@marginnote
\@ifpackageloaded{marginnote}{}{\RequirePackage{marginnote}}
\fi
%%
%    \end{macrocode}
%    \begin{macrocode}
\newcounter{@errata@erratum}
\newcounter{erratum@note}[@errata@erratum]
\if@errataOpt@record\newwrite\@errata@file
\immediate\openout\@errata@file=\jobname-errata.tex
\AtEndDocument{\closeout\@errata@file}\fi
%%
\newlength{\@errata@groupindent}
\setlength\@errata@groupindent{14.7pt}
\newcounter{@errata@uniquelistlabels}
\setcounter{@errata@uniquelistlabels}{\@errataOpt@uniquelistlabels}
\newcounter{@errata@environmenttestmode}
\setcounter{@errata@environmenttestmode}{\@errataOpt@environmenttestmode}
%    \end{macrocode}
%%
%    \begin{macrocode}
\NewDocumentCommand{\@errata@declareKey}{ O{UNDEF} m }
{\@errata@setvalue{#2/.initial={#1}}}
%%
\NewDocumentCommand{\@errata@declareStoreKey}{ O{UNDEF} m }
{%% Basically 'estore in', but must build macro name from argument
\pgfkeys{/@errata@/#2/.code={\expandafter\edef\csname @errata@#2\endcsname{##1}}}%
\pgfkeys{/@errata@/#2={#1}}%
}
%%
\NewDocumentCommand{\@errata@setcmdkeys}{ O{0} m }{%
\xintFor ##9 in {#2} \do {%
\edef\@errata@key{##9}%
\@errata@isnotempty{##9} {%
  \@errata@setvalue{cmdkeys/\@errata@key/.initial=#1}%
}}}
%%
\NewDocumentCommand{\errataSetFootFormat}{ m }{%
  \@errata@declareStoreKey[#1]{footformat}%
}
\NewDocumentCommand{\errataSetListFormat}{ m }{%
  \@errata@declareStoreKey[#1]{listformat}%
}
\@errata@declareStoreKey[type, description, location, date, id, pageref,
inlinemark, inlinechange, uselabel, label, margin, foot, record,
marginargs, usemarginnote, reported-by, actionid, action, actionargs, actiondescription,
erratumlistitem, erratumlistitemtitle, erratumlisttarget, onnewgroup, onsamegroup,
erratumlabel, erratumname,erratumnamedlink,customvalue]{formatkeys}% Defines \@errata@formatkeys
%%
\@errata@setcmdkeys{\@errata@formatkeys}
\@errata@setvalue{cmdkeys/usemarginnote/.default={0pt}}
%%
%    \end{macrocode}
% \begin{macro}{\errataSetKeyFormat}
% Macro for setting the format of a key
%    \begin{macrocode}
\NewDocumentCommand{\errataSetKeyFormat}{ O{} m m }{% PGF argument pattern, key, formated output
  \StrFmtSetKeyFormat[#1]{#2}{#3}%
}
%    \end{macrocode}
% \end{macro}
%    \begin{macrocode}
\DeclareExpandableDocumentCommand{\@errata@getformatted}{ m m m }{% key, args, macroname
  \StrFmtgetformatted{#1}{#2}{#3}%
}
\DeclareExpandableDocumentCommand{\@errata@getformattede}{ m m m }{% key, args, macroname
  \StrFmtgetformattede{#1}{#2}{#3}%
}
%% Set default formater for all keys
\SetDefaultKeylistFormat[#1]{listformats}{\@errata@formatkeys}
\SetDefaultKeylistFormat[#1]{footformats}{\@errata@formatkeys}
\SetDefaultKeylistFormat[#1]{errformats}{errfmt, eerrbeginfmt, eerrendfmt}
\errataSetKeyFormat[#1+#2]{marginfmt/marginnote}{#2}%
%%
\@errata@declareKey[{@errata@erratum}]{footnoteprefix}
\@errata@declareKey[\scshape{Erratum!}]{footnotebegin}
%%
%% Used internally to track label and label references
\@errata@declareKey{labelref}
\@errata@declareStoreKey{erratumname}
\@errata@setvalue{labelname/.gstore in=\@errata@labelname}
\@errata@setvalue{labelname={}}
\@errata@setvalue{prevlabel/.gestore in=\@errata@prevlabel}
\@errata@setvalue{prevlabel={}}
\@errata@setvalue{lastuniquelabel/.gestore in=\@errata@lastuniquelabel}
\@errata@setvalue{lastuniquelabel={}}
%%
\@errata@setvalue{lastcommandinlinevalue/.gstore in=\lastcommandinlinevalue}
\@errata@setvalue{lastcommandinlinevalue={}}
\@errata@setvalue{inlinevalue/.initial={}}
%    \end{macrocode}
%\iffalse
%    \begin{macrocode}
%</package>
%<*erratastyles>
%    \end{macrocode}
%\fi
%
% \begin{docTarget}{Errata default style definition}
%% Define the default errata formatting
% \end{docTarget}
% \begin{savecodeexample}{style-default-formats.tex}
%    \begin{macrocode}
%%
\errataset{%
  defaultkeyformats/.style={%
    %% Formats for the footnotes
    footformats/description/.setStrKeyFmt={##1}{~##1},
    footformats/action/.setStrKeyFmt={##1}{~##1},
    footformats/actiondescription/.setStrKeyFmt={##1}{~##1},
    %% Set the format for the action description for each erratum command
    actionformat/add/.setStrKeyFmt={##1}{ (added text)},
    actionformat/delete/.setStrKeyFmt={##1}{(deleted ``##1'')},
    actionformat/replace/.setStrKeyFmt={##1+##2}{ (original text was: ``##2'')},
    actionformat/note/.setStrKeyFmt={##1}{##1},
    %% Errata list format
    listformats/description/.setStrKeyFmt={##1}{~##1},
    listformats/erratumlistitemtitle/.setStrKeyFmt={##1}{%
      \@errata@hlink{\@errata@getvalue{labelref}}%
      {\protect\makebox[\@errata@groupindent][l]{\@errata@labelname}}},
    listformats/pageref/.setStrKeyFmt={##1}{Page~\pageref{##1} },
    listformats/type/.setStrKeyFmt={##1}{Type: ##1 },
    listformats/date/.setStrKeyFmt={##1}{ (Date reported: ##1)},
    listformats/reported-by/.setStrKeyFmt={##1}{Reported by: ##1},
    listformats/action/.setStrKeyFmt={##1}{~##1},
    listformats/actiondescription/.setStrKeyFmt={##1}{~##1},
    listformats/erratumlistitem/.setStrKeyFmt={##1}{\item[##1]},
    listformats/erratumlabel/.setStrKeyFmt={##1}{\@errata@labelname},
    listformats/erratumname/.setStrKeyFmt={##1}{##1},
    listformats/erratumnamedlink/.setStrKeyFmt={##1+##2}{%
      {\@errata@hlink{##1}{\protect\makebox[\@errata@groupindent][l]{##2}}}%
    },
    listformats/erratumlisttarget/.setStrKeyFmt={##1}{\@errata@htargeterratum{##1}{}},
    %% The name of the errata labels
    errformats/errfmt/.setStrKeyFmt={##1}{Err(##1)},
    errformats/eerrbeginfmt/.setStrKeyFmt={##1}{BErr(##1)},
    errformats/eerrendfmt/.setStrKeyFmt={##1}{EErr(##1)},
    %%
    actionidformat/add/.setStrKeyFmt={##1}{Addition},
    actionidformat/delete/.setStrKeyFmt={##1}{Deletion},
    actionidformat/note/.setStrKeyFmt={##1}{Note},
    actionidformat/replace/.setStrKeyFmt={##1}{Correction},
    %%
    listformats/onnewgroup/.setStrKeyFmt={##1}{},
    listformats/onsamegroup/.setStrKeyFmt={##1}{},
  },
%    \end{macrocode}
% \end{savecodeexample}
%
% \input{style-default-formats.tex}
%
%    \begin{macrocode}
defaultlistandfootstyle/.code={%
%    \end{macrocode}
%
% \begin{docTarget}{Default formatters}
%% Define the default format of the errata list and footnotes
% \end{docTarget}
% \begin{savecodeexample}{default-erratalist-format.tex}
%    \begin{macrocode}
\errataSetListFormat{+erratumlistitem, +erratumlisttarget, +description}
%    \end{macrocode}
% \end{savecodeexample}
% \input{default-erratalist-format.tex}
%
% \begin{savecodeexample}{default-footnote-format.tex}
%    \begin{macrocode}
\errataSetFootFormat{+description, +actiondescription}
%    \end{macrocode}
% \end{savecodeexample}
% \input{default-footnote-format.tex}
%    \begin{macrocode}
},
styles/default/.style={defaultkeyformats, defaultlistandfootstyle},
}
%%
%% END style: default
%%
%    \end{macrocode}
%
% \begin{docTarget}{Errata detailed style definition}
%% Define the detailed formatting
% \end{docTarget}
%    \begin{macrocode}
%%
\errataset{%
  detailedstylekeyformats/.style={%
    %% Errata list format
    listformats/description/.setStrKeyFmt={##1}{~(##1)},
    listformats/erratumlistitemtitle/.setStrKeyFmt={##1}{%
      {\@errata@hlink{\@errata@getvalue{labelref}}{%
          \protect\makebox[\@errata@groupindent][l]{\@errata@labelname}}%
        \@errata@hlinkerratum{\the@errata@erratum}{\@errata@erratumname}}},
    listformats/actiondescription/.setStrKeyFmt={##1}{\newline ##1},
    %% Foot format
    footformats/description/.setStrKeyFmt={##1}{~(##1)},
    footformats/action/.setStrKeyFmt={##1}{~##1},
    footformats/actiondescription/.setStrKeyFmt={##1}{~##1},
    %% Action format
    actionformat/replace/.setStrKeyFmt={##1+##2}{##1 (original text was: ``##2'')},
    actionformat/add/.setStrKeyFmt={##1}{##1 (added)},
    actionformat/delete/.setStrKeyFmt={##1}{##1 (deleted)},
  },
  margintcolorboxformat/.style={
    marginfmt/marginnote/.setStrKeyFmt={##1+##2}{%
      {\vspace{-4pt}%
        {\begin{tcolorbox}[size=fbox,nobeforeafter,
            colframe=red!50!white,colback=red!25!yellow!5!white,%
            leftrule=0pt,rightrule=0pt,bottomrule=1pt,toprule=1pt,
            arc=0cm, outer arc=0mm, % Remove the rounded corners
            fontupper=\scriptsize,%
            code={\pgfkeysalsofrom##1}
            ]##2\end{tcolorbox}}}},%
  },
  detailedlistandfootstyle/.code={%
    \errataSetListFormat{+erratumlistitem, +erratumlisttarget, +pageref, +action, +description, +date, +actiondescription}
    \errataSetFootFormat{+action, -description,:, +actiondescription}
  },
  detailedstylesetupcode/.code={%
    \newlength{\@errata@listindent}
    \setlength\@errata@listindent{33mm}
    \setlength\@errata@groupindent{16mm}
    \setcounter{@errata@uniquelistlabels}{3}
    \@ifpackageloaded{tcolorbox}{}{%
      \RequirePackage[most]{tcolorbox}
    }
    \AtBeginEnvironment{errata} {
      \setlist[description]{labelindent=0pt,labelwidth=\@errata@listindent,leftmargin=!}%
    }
  },
  styles/detailed/.style={detailedstylekeyformats, detailedlistandfootstyle,
    detailedstylesetupcode, margintcolorboxformat},
}
%%
%% END style: detailed
%%
%    \end{macrocode}
% \begin{docTarget}{Errata list table style definition}
%% Define the tablelist style for formatting the errata list as a table
% \end{docTarget}
%    \begin{macrocode}
%%
\NewDocumentCommand{\errataamersand}{}{&}
\NewDocumentCommand{\erratacustomhline}{}{\\\hline }
\NewDocumentCommand{\erratacustomendline}{}{\\}
\errataset{%
  liststyletablekeyformats/.style={%
    listformats/pageref/.setStrKeyFmt={##1}{\pageref{##1} },
    listformats/description/.setStrKeyFmt={##1}{##1},
    listformats/type/.setStrKeyFmt={##1}{Type: ##1},
    listformats/date/.setStrKeyFmt={##1}{ (Date reported: ##1)},
    listformats/reported-by/.setStrKeyFmt={##1}{Reported by: ##1},
    listformats/actiondescription/.setStrKeyFmt={##1}{ ##1},
    listformats/erratumlistitem/.setStrKeyFmt={##1}{},
    listformats/erratumlistitemtitle/.setStrKeyFmt={##1+##2}{%
      {\@errata@hlink{##1}{\protect\makebox[\@errata@groupindent][l]{##2}}}},
    listformats/onnewgroup/.setStrKeyFmt={##1}{\noexpand\erratacustomhline},
    listformats/onsamegroup/.setStrKeyFmt={##1}{\noexpand\erratacustomendline},
    listformats/erratalistbegin/.setStrKeyFmt={##1}{\makeatletter%
        \begin{longtable}{ c c c p{6cm} c }%
           & Title & Type & Action Desc & Page \\\hline%
          },
    listformats/erratalistend/.setStrKeyFmt={##1}{\end{longtable}\makeatother},
  },
  renewerrataenvfortable/.code={%
    \AtBeginDocument{%
      \renewenvironment{errata}{%
        \@errata@getformatted{listformats/erratalistbegin}{}{\@errata@tmp@erratalistbegin}%
        \@errata@tmp@erratalistbegin%
      }{\@errata@getformatted{listformats/erratalistend}{}{\@errata@tmp@erratalistend}%
        \@errata@tmp@erratalistend%
      }}
  },
  liststylefortable/.code={
    \errataSetListFormat{+erratumlistitemtitle,+erratumlisttarget,\noexpand\errataamersand,%
      +erratumnamedlink,\noexpand\errataamersand,+action,\noexpand\errataamersand,%
      +actiondescription,\noexpand\errataamersand,+pageref}%
  },
  styles/tablelist/.style={liststyletablekeyformats, renewerrataenvfortable, liststylefortable},
}
%%
%% END style tablelist
%%
%    \end{macrocode}
%\iffalse
%    \begin{macrocode}
%</erratastyles>
%<*package>
%    \end{macrocode}
%\fi
%    \begin{macrocode}
%%
%% This is file `erratastyles.tex',
%% generated with the docstrip utility.
%%
%% The original source files were:
%%
%% errata.dtx  (with options: `erratastyles')
%% 
%% Define the default errata formatting
%%
\errataset{%
  defaultkeyformats/.style={%
    %% Formats for the footnotes
    footformats/description/.setStrKeyFmt={##1}{~##1},
    footformats/action/.setStrKeyFmt={##1}{~##1},
    footformats/actiondescription/.setStrKeyFmt={##1}{~##1},
    %% Set the format for the action description for each erratum command
    actionformat/add/.setStrKeyFmt={##1}{ (added text)},
    actionformat/delete/.setStrKeyFmt={##1}{(deleted ``##1'')},
    actionformat/replace/.setStrKeyFmt={##1+##2}{ (original text was: ``##2'')},
    actionformat/note/.setStrKeyFmt={##1}{##1},
    %% Errata list format
    listformats/description/.setStrKeyFmt={##1}{~##1},
    listformats/erratumlistitemtitle/.setStrKeyFmt={##1}{%
      \@errata@hlink{\@errata@getvalue{labelref}}%
      {\protect\makebox[\@errata@groupindent][l]{\@errata@labelname}}},
    listformats/pageref/.setStrKeyFmt={##1}{Page~\pageref{##1} },
    listformats/type/.setStrKeyFmt={##1}{Type: ##1 },
    listformats/date/.setStrKeyFmt={##1}{ (Date reported: ##1)},
    listformats/reported-by/.setStrKeyFmt={##1}{Reported by: ##1},
    listformats/action/.setStrKeyFmt={##1}{~##1},
    listformats/actiondescription/.setStrKeyFmt={##1}{~##1},
    listformats/erratumlistitem/.setStrKeyFmt={##1}{\item[##1]},
    listformats/erratumlabel/.setStrKeyFmt={##1}{\@errata@labelname},
    listformats/erratumname/.setStrKeyFmt={##1}{##1},
    listformats/erratumnamedlink/.setStrKeyFmt={##1+##2}{%
      {\@errata@hlink{##1}{\protect\makebox[\@errata@groupindent][l]{##2}}}%
    },
    listformats/erratumlisttarget/.setStrKeyFmt={##1}{\@errata@htargeterratum{##1}{}},
    %% The name of the errata labels
    errformats/errfmt/.setStrKeyFmt={##1}{Err(##1)},
    errformats/eerrbeginfmt/.setStrKeyFmt={##1}{BErr(##1)},
    errformats/eerrendfmt/.setStrKeyFmt={##1}{EErr(##1)},
    %%
    actionidformat/add/.setStrKeyFmt={##1}{Addition},
    actionidformat/delete/.setStrKeyFmt={##1}{Deletion},
    actionidformat/note/.setStrKeyFmt={##1}{Note},
    actionidformat/replace/.setStrKeyFmt={##1}{Correction},
    %%
    listformats/onnewgroup/.setStrKeyFmt={##1}{},
    listformats/onsamegroup/.setStrKeyFmt={##1}{},
  },
defaultlistandfootstyle/.code={%
%% Define the default format of the errata list and footnotes
\errataSetListFormat{+erratumlistitem, +erratumlisttarget, +description}
\errataSetFootFormat{+description, +actiondescription}
},
styles/default/.style={defaultkeyformats, defaultlistandfootstyle},
}
%%
%% END style: default
%%
%% Define the detailed formatting
%%
\errataset{%
  detailedstylekeyformats/.style={%
    %% Errata list format
    listformats/description/.setStrKeyFmt={##1}{~(##1)},
    listformats/erratumlistitemtitle/.setStrKeyFmt={##1}{%
      {\@errata@hlink{\@errata@getvalue{labelref}}{%
          \protect\makebox[\@errata@groupindent][l]{\@errata@labelname}}%
        \@errata@hlinkerratum{\the@errata@erratum}{\@errata@erratumname}}},
    listformats/actiondescription/.setStrKeyFmt={##1}{\newline ##1},
    %% Foot format
    footformats/description/.setStrKeyFmt={##1}{~(##1)},
    footformats/action/.setStrKeyFmt={##1}{~##1},
    footformats/actiondescription/.setStrKeyFmt={##1}{~##1},
    %% Action format
    actionformat/replace/.setStrKeyFmt={##1+##2}{##1 (original text was: ``##2'')},
    actionformat/add/.setStrKeyFmt={##1}{##1 (added)},
    actionformat/delete/.setStrKeyFmt={##1}{##1 (deleted)},
  },
  margintcolorboxformat/.style={
    marginfmt/marginnote/.setStrKeyFmt={##1+##2}{%
      {\vspace{-4pt}%
        {\begin{tcolorbox}[size=fbox,nobeforeafter,
            colframe=red!50!white,colback=red!25!yellow!5!white,%
            leftrule=0pt,rightrule=0pt,bottomrule=1pt,toprule=1pt,
            arc=0cm, outer arc=0mm, % Remove the rounded corners
            fontupper=\scriptsize,%
            code={\pgfkeysalsofrom##1}
            ]##2\end{tcolorbox}}}},%
  },
  detailedlistandfootstyle/.code={%
    \errataSetListFormat{+erratumlistitem, +erratumlisttarget, +pageref, +action, +description, +date, +actiondescription}
    \errataSetFootFormat{+action, -description,:, +actiondescription}
  },
  detailedstylesetupcode/.code={%
    \newlength{\@errata@listindent}
    \setlength\@errata@listindent{33mm}
    \setlength\@errata@groupindent{16mm}
    \setcounter{@errata@uniquelistlabels}{3}
    \@ifpackageloaded{tcolorbox}{}{%
      \RequirePackage[most]{tcolorbox}
    }
    \AtBeginEnvironment{errata} {
      \setlist[description]{labelindent=0pt,labelwidth=\@errata@listindent,leftmargin=!}%
    }
  },
  styles/detailed/.style={detailedstylekeyformats, detailedlistandfootstyle,
    detailedstylesetupcode, margintcolorboxformat},
}
%%
%% END style: detailed
%%
%% Define the tablelist style for formatting the errata list as a table
%%
\NewDocumentCommand{\errataamersand}{}{&}
\NewDocumentCommand{\erratacustomhline}{}{\\\hline }
\NewDocumentCommand{\erratacustomendline}{}{\\}
\errataset{%
  liststyletablekeyformats/.style={%
    listformats/pageref/.setStrKeyFmt={##1}{\pageref{##1} },
    listformats/description/.setStrKeyFmt={##1}{##1},
    listformats/type/.setStrKeyFmt={##1}{Type: ##1},
    listformats/date/.setStrKeyFmt={##1}{ (Date reported: ##1)},
    listformats/reported-by/.setStrKeyFmt={##1}{Reported by: ##1},
    listformats/actiondescription/.setStrKeyFmt={##1}{ ##1},
    listformats/erratumlistitem/.setStrKeyFmt={##1}{},
    listformats/erratumlistitemtitle/.setStrKeyFmt={##1+##2}{%
      {\@errata@hlink{##1}{\protect\makebox[\@errata@groupindent][l]{##2}}}},
    listformats/onnewgroup/.setStrKeyFmt={##1}{\noexpand\erratacustomhline},
    listformats/onsamegroup/.setStrKeyFmt={##1}{\noexpand\erratacustomendline},
    listformats/erratalistbegin/.setStrKeyFmt={##1}{\makeatletter%
        \begin{longtable}{ c c c p{6cm} c }%
           & Title & Type & Action Desc & Page \\\hline%
          },
    listformats/erratalistend/.setStrKeyFmt={##1}{\end{longtable}\makeatother},
  },
  renewerrataenvfortable/.code={%
    \AtBeginDocument{%
      \renewenvironment{errata}{%
        \@errata@getformatted{listformats/erratalistbegin}{}{\@errata@tmp@erratalistbegin}%
        \@errata@tmp@erratalistbegin%
      }{\@errata@getformatted{listformats/erratalistend}{}{\@errata@tmp@erratalistend}%
        \@errata@tmp@erratalistend%
      }}
  },
  liststylefortable/.code={
    \errataSetListFormat{+erratumlistitemtitle,+erratumlisttarget,\noexpand\errataamersand,%
      +erratumnamedlink,\noexpand\errataamersand,+action,\noexpand\errataamersand,%
      +actiondescription,\noexpand\errataamersand,+pageref}%
  },
  styles/tablelist/.style={liststyletablekeyformats, renewerrataenvfortable, liststylefortable},
}
%%
%% END style tablelist
%%
\endinput
%%
%% End of file `erratastyles.tex'.

\errataset{.errataapplystyles={\@errataOpt@styles}}
%%
\DeclareExpandableDocumentCommand{\@errata@htargetname}{ m } {errata#1}%
\DeclareExpandableDocumentCommand{\@errata@htargetlisttname}{ m } {erratalist#1}%
\NewDocumentCommand{\@errata@hlink}{ m m } {#2}%
\NewDocumentCommand{\@errata@label}{ m } {\label{#1}}%
\NewDocumentCommand{\@errata@htargeterratum}{ O{} m m } {\@errata@label{#1}#2}
\NewDocumentCommand{\@errata@hlinkerratum}{ m m } {\@errata@hlink{\@errata@htargetname{#1}}{#2}}
\NewDocumentCommand{\@errata@hlinkerratumlist}{ m m } {\@errata@hlink{\@errata@htargetlisttname{#1}}{#2}}
%%
%% Enable internal linking if hyperref
\if@errataOpt@hyperref
\RenewDocumentCommand{\@errata@hlink}{ m m } {%
\protected@edef\@errata@expanded@arg{#1}%
\hyperlink{\@errata@expanded@arg}{#2}}%
%%
%% \@errata@htargeterratum
%% ARG1: Optional argument to indicate whether to add labels,
%% ARG2: The target name of the primary label/hyperlink
%% ARG3: The target name of the optional custom label
%% ARG4: The text to insert after the label/hypertarget
%% Hypertarget creates a hyper-link anchor, used to link to this location with \hyperlink
%% The label is needed for such as ref/pageref/cref
\RenewDocumentCommand{\@errata@htargeterratum}{ O{} O{} m m }%
{\Hy@raisedlink{\hypertarget{#3}{}}%
\@errata@isempty{#1}{% Add labels
\@errata@label{#3}% Primary label
\@errata@isnotempty{#2}{\@errata@label{#2}}% Seconday custom label
}{#4}}%
\fi
%%

%    \end{macrocode}
% \begin{docTarget}{Cleveref support}
%% Code for cleveref support
% \end{docTarget}
%    \begin{macrocode}
\if@errataOpt@cleveref
  \AtBeginDocument{%
    \Crefname{errata}{Err}{Errata}
    \crefname{errata}{err}{errata}
    \crefformat{errata}{Err(#2#1#3)}
    \Crefformat{errata}{Err(#2#1#3)}
  }
  \RenewDocumentCommand{\@errata@label}{ m } {%
    \protected@edef\@errata@expanded@arg{#1}\label[errata]{\@errata@expanded@arg}}%
\fi
%%
\NewDocumentCommand{\@errata@FormatKeywords}{ O{} m m m }{% key, keys/values, format-string
\@StrFmt@FormatKeywords[#1]{#2}{cmdkeys}{#3}{#4}%
}
%%
\newcounter{@errata@lowerlabellim}
\setcounter{@errata@lowerlabellim}{0}
%% Return true of the label name can be set
\newcommand\ifuniquelabeltest[1]{%
%% Current document class doesn't have chapters.
    \setcounter{@errata@lowerlabellim}{#1}%
  \@ifundefined{thechapter}{%
    \stepcounter{@errata@lowerlabellim}%
  }{}%
  \ifboolexpr{test {\ifnumcomp{\value{@errata@uniquelistlabels}}{=}{0}} or test {\ifnumcomp{\value{@errata@uniquelistlabels}}{>}{\value{@errata@lowerlabellim}}}}}
%%
\DeclareExpandableDocumentCommand{\@errata@setlabelref}{ } {%
\@ifundefined{thechapter}{}{%
  \ifnumcomp{\value{chapter}}{>}{0}{%
    \@errata@setvalue{labelname=\thechapter}%
    \@errata@setvalue{labelref={chapter.\thechapter}}%
  }{}}%
%% Check section
\@ifundefined{thesection}{}{%
  \ifnumcomp{\value{section}}{>}{0}{%
    \@errata@setvalue{labelref={section.\thesection}}%
    \ifboolexpr{test {\ifuniquelabeltest{1}} }{\@errata@setvalue{labelname=\thesection}}{}%
    %% Check subsection
    \@ifundefined{thesubsection}{}{%
      \ifnumcomp{\value{subsection}}{>}{0}{%
        \@errata@setvalue{labelref={subsection.\thesubsection}}%
        \ifboolexpr{test {\ifuniquelabeltest{2}} }{\@errata@setvalue{labelname=\thesubsection}}{}%
        %% Check subsubsection
        \@ifundefined{thesubsubsection}{}{%
          \ifnumcomp{\value{subsubsection}}{>}{0}{%
            \@errata@setvalue{labelref={subsubsection.\thesubsubsection}}%
            \ifboolexpr{test {\ifuniquelabeltest{3}} }{\@errata@setvalue{labelname=\thesubsubsection}}{}%
          }{}%
        }%
      }{}% END subsubsection
    }% END subsection
  }{}%
}% END section
}
%%
%    \end{macrocode}
% \begin{macro}{\erratumItem}
% Macro for adding an item to the errata list
%    \begin{macrocode}
%% ARG1: Optional value to add to the end
%% ARG2: Item title
%% ARG2: errata-format-text
\DeclareDocumentCommand{\erratumItem}{ O{} m m } {%
\@errata@getformatted{listformats/erratumlistitem}{#2}{\@errata@tmplistitem}%
\@errata@tmplistitem#3#1%
}
%%
%    \end{macrocode}
% \end{macro}
%    \begin{macrocode}
%    \end{macrocode}
% \begin{macro}{\printerrata}
% Macro for inserting an existing errata list into the current document.
%    \begin{macrocode}
\DeclareDocumentCommand{\printerrata}{ O{-errata.tex} m } {%
\IfFileExists{#2#1}{%
\begin{errata}\input{#2#1}\end{errata}}{}%
}
%    \end{macrocode}
% \end{macro}
%%
% \begin{macro}{\PrintErrata}
% Macro for inserting the errata list produced for the current document.
%    \begin{macrocode}
\DeclareDocumentCommand{\PrintErrata}{} {%
\if@errataOpt@record\immediate\closeout\@errata@file\fi%
\printerrata{\jobname}%
}
%    \end{macrocode}
% \end{macro}
%    \begin{macrocode}
%%
\newcommand\ifdomargin[0]{
  \ifboolexpr{ bool{@errataOpt@margins} and test{\ifthenelse{\equal{\@errata@getvalue{cmdkeys/margin}}{true}}}}}
\newcommand\ifdofoot[0]{
  \ifboolexpr{ bool{@errataOpt@foots} and test{\ifthenelse{\equal{\@errata@getvalue{cmdkeys/foot}}{true}}}}}
\newcommand\ifdorecord[0]{
  \ifboolexpr{ bool{@errataOpt@record} and test{\ifthenelse{\equal{\@errata@getvalue{cmdkeys/record}}{true}}}}}
\newenvironment{errata}{\makeatletter%
\begin{description}}{\end{description}%
\makeatother
}
%%
%    \end{macrocode}
% \begin{environment}{erratum}
% The erratum environment used for creating an extended errata.
%    \begin{macrocode}
\DeclareDocumentEnvironment{erratum}{ O{} m }% keys, explanation
{\@erratum@cmd{}{#1}{action={Long errata}, description={#2}}%
\expandafter\@errata@htargeterratum{\@errata@htargetname{\the@errata@erratum}}{}%
\if@errataOpt@foots%
\footnotetext[\value{@errata@erratum}]{{\@errata@getvalue{footnotebegin} #2}}%
\fi%
\protected@edef\@errata@tmp@marginargs{\@errata@getvalue{cmdkeys/marginargs}}%
\ifboolexpr{test {\ifdomargin} }%
{\@errata@getformatted{errformats/eerrbeginfmt}{\@errata@hlinkerratumlist{\the@errata@erratum}{\arabic{@errata@erratum}}}{\@errata@tmpfmtvalue}%
\@errata@marginnote[\@errata@tmp@marginargs]{\@errata@tmpfmtvalue}%
}{}%
\@errata@dorecord[\\]{listformats}{\@errata@listformat}{\@errata@argsstring}%
\def\eAdd##1{[##1]$_a$}%
\def\eDelete##1{\erratum@mark[]$_d^{\@thefnmark}$\if@errataOpt@foots\@footnotetext{deleted: ##1}\fi}%
\def\eReplace##1##2{\erratum@mark[##2]$_r^{\@thefnmark}$\if@errataOpt@foots\@footnotetext{was: ##1}\fi}
\ignorespaces}%
{\ifboolexpr{test {\ifdomargin} }%
{\@errata@getformatted{errformats/eerrendfmt}{\@errata@hlinkerratumlist{\the@errata@erratum}{\arabic{@errata@erratum}}}{\@errata@tmpfmtvalue}%
\@errata@marginnote[\@errata@tmp@marginargs]{\@errata@tmpfmtvalue}%
}{}%
}%
%    \end{macrocode}
% \end{environment}
%    \begin{macrocode}
%%
\def\erratum@mark{\stepcounter{erratum@note}{}
\def\@thefnmark{\arabic{@errata@erratum}:\arabic{erratum@note}}}
%    \end{macrocode}
% \begin{macro}{\erratatemessage}
%% Write info to log file
%    \begin{macrocode}
\NewDocumentCommand{\erratamessage}{ m } {% erratakeyvalues
\ifnum\value{@errata@erratum}>0\typeout{}%
\typeout{This document contains \arabic{@errata@erratum} Errata; see \jobname-errata.tex!}%
\typeout{}%
\fi}
%    \end{macrocode}
% \end{macro}
%
%%
% \begin{macro}{\@errata@dorecord}
%    \begin{macrocode}
\NewDocumentCommand{\@errata@dorecord}{ O{} m m m } {% formattype, listformat, argsstring
\ifboolexpr{ test {\ifdorecord} }{%
%% Get the formatted code for a new group
  \ifboolexpr{ not test {\ifthenelse{\equal{\@errata@getvalue{cmdkeys/onnewgroup}}{}}} }%
  {%
    \@errata@getformatted{listformats/onnewgroup}{\@errata@labelname}{\@errata@tmp@ongroup}%
  }%
  {\@errata@getformatted{listformats/onsamegroup}{\@errata@labelname}{\@errata@tmp@ongroup}}%
\@errata@FormatKeywords[errata@formatted@output]{#2}{#4}{#3}%
\@errata@record[\@errata@tmp@ongroup]{\expandonce{\errata@formatted@output}}%
}{}%
}
%    \end{macrocode}
% \end{macro}
% \begin{macro}{\@errata@record}
%% Create an entry in the errata list produced by PrintErrata
%    \begin{macrocode}
\NewDocumentCommand{\@errata@record}{ O{} m } {% erratakeyvalues
\@errata@getformatted{errformats/errfmt}{\arabic{@errata@erratum}}{\@errata@erratumname}%
\@errata@setvalue{cmdkeys/erratumlistitemtitle/.get=\@errata@tmp@itemtitle}%
\@errata@getformattede{listformats/erratumlistitemtitle}{\@errata@tmp@itemtitle}{\@errata@tmp@formatteditemtitle}%
\protected@write\@errata@file{}%
{\string\erratumItem[#1]{\@errata@tmp@formatteditemtitle}{#2}}%
}
%    \end{macrocode}
% \end{macro}
%    \begin{macrocode}
\newif\if@errata@inbadenvironment
\newif\if@errata@inknownbadenvironment
%%
\NewDocumentCommand{\@erratum@checkenvironment}{}{%
\@errata@inknownbadenvironmentfalse%
\@errata@inbadenvironmentfalse%
%%
\ifinner%
\@errata@inknownbadenvironmenttrue%
\fi%
\ifmmode%
\@errata@inknownbadenvironmenttrue%
\fi%
\ifthenelse{\equal{\@currenvir}{tcb@savebox}}%
{\@errata@inknownbadenvironmenttrue}%
{}%
%%
\ifnumgreater{\value{@errata@environmenttestmode}}{0}%
{% At least 1, so lets consider any environment with \@captype dangerous
\ifx\@captype\@undefined%
\else%
\@errata@inbadenvironmenttrue%
\fi%
\ifnumequal{\value{@errata@environmenttestmode}}{2}%
{%
\ifthenelse{\equal{\@currenvir}{document}}%
{}{\@errata@inbadenvironmenttrue}%
}{}% End == 2
}{}% End > 0
}
\edef\emptyval{}%
%% Create the margin note
\NewDocumentCommand{\@errata@marginnote}{ O{\emptyval} m }{%
%% We use marginpar if we can, as it handles stacking multiple values vertically
\let\@errata@marginnote@cmd\marginpar%
%% If captype is defined, we're probably inside a float
\if@errata@inknownbadenvironment%
\let\@errata@marginnote@cmd\marginnote%
\fi
\if@errata@inbadenvironment%
\let\@errata@marginnote@cmd\marginnote%
\fi
%% If user provided argument 'usemarginnote', we use \marginnote macro
\protected@edef\@errata@tmp@marginshift{\@errata@getvalue{cmdkeys/usemarginnote}}%
\@errata@isnotempty{\@errata@tmp@marginshift}{%
  \let\@errata@marginnote@cmd\marginnote%
  \renewcommand*{\marginnotevadjust}{\@errata@tmp@marginshift}%
}%
\@errata@getformatted{marginfmt/marginnote}{{#1}+{#2}}{\@errata@tmp@marginfmtvalue}%
\@errata@marginnote@cmd{\@errata@tmp@marginfmtvalue}%
}
%% Do the pagelayout of the errata (inline changes, margin, and footnote)
\NewDocumentCommand{\@errata@dolayoutonpage}{ m m m m }{% type(a,r,d), original, change, errata-formatted-text
\ifboolexpr{test {\ifdomargin} }%
{\@errata@getformatted{errformats/errfmt}{\@errata@hlinkerratumlist{\the@errata@erratum}{\arabic{@errata@erratum}}}{\@errata@tmpfmtvalue}%
  \protected@edef\@errata@tmp@marginargs{\@errata@getvalue{cmdkeys/marginargs}}%
  \@errata@marginnote[\@errata@tmp@marginargs]{\@errata@tmpfmtvalue}%
}{}%
%% Handle inline changes
\ifboolexpr{ bool{@errataOpt@inlinechanges} and
  test {\ifthenelse{\equal{\@errata@getvalue{cmdkeys/inlinechange}}{true}}} }%
{% Make inline change
%%\@errata@setvalues{lastcommandinlinevalue={#3}, inlinevalue={#3}}
  \@errata@setvalue{lastcommandinlinevalue={#3}}%
  \@errata@setvalue{inlinevalue={#3}}%
}%
{% Use original value
  \@errata@setvalue{lastcommandinlinevalue={#2}}%
  \@errata@setvalue{inlinevalue={#2}}%
}%
%% If trying to set lastcommandinlinevalue=inlinevalue here with
%% (\@errata@setvalue{lastcommandinlinevalue=\@errata@getvalue{inlinevalue}}%),
%% it adds an unwanted space to lastcommandinlinevalue for some reason. Therefore,
%% both lastcommandinlinevalue and inlinevalue are set to the same value separately.
%%
\ifboolexpr{ bool{@errataOpt@inlinemarks} and
  test {\ifthenelse{\equal{\@errata@getvalue{cmdkeys/inlinemark}}{true}}} }%
{% Use add inlinemark
  \@errata@setvalue{lastcommandinlinevalue={[]}}%
  \@errata@setvalue{inlinevalue/.prefix={[}}%
  \@errata@setvalue{inlinevalue/.append={]\ensuremath{_#1^{\arabic{@errata@erratum}}}}}%
}{}%
\protected@edef\@errata@tmp@dolabel{}%
\ifmmode% In math mode, we can't add a label
\eappto\@errata@tmp@dolabel{nolabel}%
\fi%
\@errata@htargeterratum[\@errata@tmp@dolabel][\@errata@getvalue{cmdkeys/label}]{\@errata@htargetname{\the@errata@erratum}}{\@errata@getvalue{inlinevalue}}%
%%
\ifboolexpr{test {\ifdofoot} }{%
  \protected@edef\@errata@tmp@footprefix{\arabic{\@errata@getvalue{footnoteprefix}}}% Expand the formatting code here, and not inside the footnote macro
  \expandafter\footnotetext\expandafter[\@errata@tmp@footprefix]%
  {{\@errata@getvalue{footnotebegin}}%
    \@errata@isnotempty{#4}{#4}}%
}{}%
}%
%%
\NewDocumentCommand{\@erratum@cmdinit}{}{%
\@erratum@setprevispace% Handle if there was a previous white space glue
\@erratum@checkenvironment%
}
%% ARG1: cmddefaultargs string
%% ARG2: Key/Args from cmd
%% ARG3: extra Key/Args
\NewDocumentCommand{\@erratum@cmd}{ m m m } {%
\refstepcounter{@errata@erratum}%
\@errata@setcmdkeys[]{\@errata@formatkeys}% Reset all values
%% Apply the default args
\@errata@setvalues[cmdkeys/]{defaultargs, #1}%
%%
\@errata@setvalue{cmdkeys/uselabel={\@errata@htargetname{\the@errata@erratum}}}%
\@errata@getformatted{errformats/errfmt}{\arabic{@errata@erratum}}{\@errata@erratumname}%
\@errata@setvalue{cmdkeys/erratumname/.expand once=\@errata@erratumname}%
%%
\@errata@setvalues[cmdkeys/]{#2, #3}%
\@errata@setvalues{cmdkeys/actionid/.get=\@errata@tmp@actionid}%
%% Set 'action' and 'actiondescription' based on actionid
\@errata@isnotempty{\@errata@tmp@actionid}{%
  \@errata@setvalues{cmdkeys/actionargs/.get=\@errata@tmp@actionargs}%
  \@errata@getformattede{actionformat/\@errata@tmp@actionid}{\@errata@tmp@actionargs}{\@errata@tmpfmtvalue}%
  \@errata@setvalues[cmdkeys/]{actiondescription/.expanded={\@errata@tmpfmtvalue}}%
  \@errata@getformattede{actionidformat/\@errata@tmp@actionid}{}{\@errata@tmpfmtvalue}%
  \@errata@setvalues[cmdkeys/]{action/.expanded={\@errata@tmpfmtvalue}}%
}%
%% Define macro for the args string
\protected@edef\@errata@argsstring{#2}%
\@errata@isnotempty{\@errata@argsstring}{%
  \eappto\@errata@argsstring{, }%
}%
\@errata@setvalue{prevlabel=\@errata@labelname}%
\@errata@setlabelref{}%
\ifboolexpr{ test {\ifdorecord} }{%
  %% Compare current label with previous. Print only if not already printed
  \ifnumcomp{\value{@errata@uniquelistlabels}}{>}{0}%
  {% True
    \ifthenelse{\expandafter\equal{\@errata@lastuniquelabel}{\@errata@labelname}}%
    {\@errata@setvalue{labelname={}}% Hide the label, if it equals the previous label
      \@errata@setvalue{cmdkeys/onnewgroup={}}%
    }%
    {\@errata@setvalue{prevlabel/.expanded=\@errata@labelname}%
      \@errata@setvalue{cmdkeys/onnewgroup/.expanded=\@errata@labelname}%
    }%
    \@errata@setvalue{lastuniquelabel/.expanded=\@errata@labelname}%
  }{\@errata@setvalue{prevlabel=\@errata@labelname}}%
}{}%
%%
\@errata@setvalue{cmdkeys/erratumlistitemtitle={{\@errata@getvalue{labelref}}+{\@errata@labelname}}}%
\@errata@setvalue{cmdkeys/erratumnamedlink/.expand once={{\@errata@getvalue{labelref}}+{\@errata@erratumname}}}%
\protected@eappto\@errata@argsstring{#3}%
%% Add action description
\protected@eappto\@errata@argsstring{, actiondescription={\@errata@getvalue{cmdkeys/actiondescription}}}%
\protected@eappto\@errata@argsstring{, action={\@errata@getvalue{cmdkeys/action}}}%
%% Handle pageref reference
\@errata@isnotempty{\@errata@getvalue{cmdkeys/uselabel}}{%
  \protected@eappto\@errata@argsstring{, pageref={\@errata@getvalue{cmdkeys/uselabel}}}%
}%
\protected@eappto\@errata@argsstring{, erratumlistitemtitle={\@errata@getvalue{cmdkeys/erratumlistitemtitle}}}%
\protected@eappto\@errata@argsstring{, erratumnamedlink={\@errata@getvalue{cmdkeys/erratumnamedlink}}}%
%% Set the link target for this errata in the errata list (erratalist<number>)
\xappto\@errata@argsstring{, erratumlisttarget={\@errata@htargetlisttname{\the@errata@erratum}}}%
}
%%
%% Macros for handling spacing
%%
\newcommand\ifstrequalexpand{\expandafter\ifstrequal\expandafter}%
%%
%% Handle if space after macro call should be ignored
\NewDocumentCommand{\@erratum@handleprevisspace}{}{%
\ifhmode%
\ifboolexpr{ bool {@errata@prevspace} and test {\ifstrequalexpand{\lastcommandinlinevalue}{}} }{%
\ignorespaces}{}%
\fi%
}
%%
\newif\if@errata@prevspace
%%
\NewDocumentCommand{\@erratum@setprevispace}{}{%
\ifhmode%
\ifdim \lastskip=\z@%
\@errata@prevspacefalse%
\else%
\@errata@prevspacetrue%
\fi%
\fi%
}
%    \end{macrocode}
%%
%% User commands for registering errata changes
% \begin{macro}{\erratumAdd}
%%
%% Macro for creating an Addition
%%
%    \begin{macrocode}
\NewDocumentCommand{\@erratum@Add}{ O{} m }{% keyvals, new
\@erratum@cmdinit%
{% Group is necessary here to handle cref reference format for label type `@errata@erratum'.
%% \@erratum@cmd must be within the group, as pgfkeys are group local
\@erratum@cmd{defaultargsadd}{#1}{actionid={add}, actionargs={#2}}%
\@errata@dolayoutonpage{a}{}{#2}{\@errata@FormatKeywords{footformats}{\@errata@argsstring}{\@errata@footformat}}%
\@errata@dorecord[\\]{listformats}{\@errata@listformat}{\@errata@argsstring}%
}%
%% Handle if the space after the macro call should be ignored
\@erratum@handleprevisspace%
}
%    \end{macrocode}
% \end{macro}
%
% \begin{macro}{\erratumReplace}
%%
%% Macro for creating a Replacement
%%
%    \begin{macrocode}
\NewDocumentCommand{\@erratum@Replace}{ O{} m m }{% keyvals, old, new
\@erratum@cmdinit%
{% Group is necessary here to handle cref reference format for label type `@errata@erratum'.
%% \@erratum@cmd must be within the group, as pgfkeys are group local
\@erratum@cmd{defaultargsreplace}{#1}{actionid={replace}, actionargs={#3+#2}}%
\@errata@dolayoutonpage{r}{#2}{#3}{\@errata@FormatKeywords{footformats}{\@errata@argsstring}{\@errata@footformat}}%
\@errata@dorecord[\\]{listformats}{\@errata@listformat}{\@errata@argsstring}%
}%
%% Handle if the space after the macro call should be ignored
\@erratum@handleprevisspace%
}
%    \end{macrocode}
% \end{macro}
% \begin{macro}{\erratumDelete}
%%
%% Macro for creating a Deletion
%%
%    \begin{macrocode}
\NewDocumentCommand{\@erratum@Delete}{ O{} m }{% keyvals, removed
\@erratum@cmdinit%
{% Group is necessary here to handle cref reference format for label type `@errata@erratum'.
%% \@erratum@cmd must be within the group, as pgfkeys are group local
\@erratum@cmd{defaultargsdelete}{#1}{actionid={delete}, actionargs={#2}}%
\@errata@dolayoutonpage{d}{#2}{}{\@errata@FormatKeywords{footformats}{\@errata@argsstring}{\@errata@footformat}}%
\@errata@dorecord[\\]{listformats}{\@errata@listformat}{\@errata@argsstring}%
}%
\@erratum@handleprevisspace%
}
%    \end{macrocode}
% \end{macro}
% \begin{macro}{\erratumNote}
%%
%% Macro for creating a Note entry
%%
%    \begin{macrocode}
\NewDocumentCommand{\@erratum@Note}{ O{} m }{% keyvals, message
\@erratum@cmdinit%
{% Group is necessary here to handle cref reference format for label type `@errata@erratum'.
%% \@erratum@cmd must be within the group, as pgfkeys are group local
\@erratum@cmd{defaultargsnote}{#1}{actionid={note}, actionargs={#2}}%
\@errata@dolayoutonpage{n}{}{#2}{\@errata@FormatKeywords{footformats}{\@errata@argsstring}{\@errata@footformat}}%
\@errata@dorecord[\\]{listformats}{\@errata@listformat}{\@errata@argsstring}%
}%
%% Handle if the space after the macro call should be ignored
\@erratum@handleprevisspace%
}
%    \end{macrocode}
% \end{macro}
%    \begin{macrocode}
%% Define with mandatory description argument if the requirecmddescription option is true
\if@errataOpt@requirecmddescription
\NewDocumentCommand{\erratumAdd}{ O{} m m }{% keyvals, description, new
  \@erratum@Add[#1,description={#2}]{#3}}%
\NewDocumentCommand{\erratumReplace}{ O{} m m m }{% keyvals, description, old, new
\@erratum@Replace[#1,description={#2}]{#3}{#4}}%
\NewDocumentCommand{\erratumDelete}{ O{} m m }{% keyvals, description, removed
  \@erratum@Delete[#1,description={#2}]{#3}}%
\NewDocumentCommand{\erratumNote}{ O{} m m }{% keyvals, description, note
  \@erratum@Note[#1,description={#2}]{#3}}%
%% And define without description if the requirecmddescription option is false
\else
\let\erratumDelete\@erratum@Delete
\let\erratumAdd\@erratum@Add
\let\erratumReplace\@erratum@Replace
\let\erratumNote\@erratum@Note
\fi
%    \end{macrocode}
%\iffalse
%    \begin{macrocode}
%</package>
%    \end{macrocode}
%\fi
% \newpage\section{The Errata of this Document}
% \PrintErrata
% \newpage
% \PrintIndex
% \Finale
%\iffalse
\endinput
% LocalWords:  alltogether erratamessage erratamessage thebook.tex theerrata.tex thebook
% LocalWords:  newpage ifmargins iffoots newif marginsfalse footsfalse ifrecord newwrite
% LocalWords:  recordfalse marginstrue footstrue recordtrue newcounter openout ifnum
% LocalWords:  subsubsection newenvironment newcommand keyvals setkeys stepcounter texttt
% LocalWords:  marginpar typeout scshape theerratum ednotemessage texttt nomark nomargins
% LocalWords:  Schroll-Fleischer nofoots colors ifmark ignorespaces doctex
%%% Local Variables:
%%% fill-column: 79
%%% TeX-master: "errata.dtx"
%%% End:
%\fi
