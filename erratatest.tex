%\documentclass[a4paper,12pt]{memoir}
\documentclass[a4paper,12pt]{report}
\usepackage[T1]{fontenc}
\usepackage[utf8]{inputenc}
\usepackage[margin=1.6in,marginparsep=3mm]{geometry}

\usepackage{pgfkeys}
\usepackage{caption}
\usepackage{etoolbox}

\usepackage{etextools}

\setkeys{caption}{debug=0}
\PassOptionsToPackage{log-declarations=false}{xparse}

\makeatletter
\newcommand\IfPackageLoaded[3]{\@ifpackageloaded{#1}{#2}{#3}}%
\makeatother

\usepackage{lineno}

\IfPackageLoaded{lineno}{
\makeatletter
% \patchcmd{<cmd>}{<search>}{<replace>}{<success>}{<failure>}
%\patchcmd{\@startsection}{\@ifstar}{\nolinenumbers\@ifstar}{}{}
%\patchcmd{\@xsect}{\ignorespaces}{\linenumbers\ignorespaces}{}{}
\makeatother
\setpagewiselinenumbers
\linenumbers
\modulolinenumbers[100]
}{
\newif\ifLineNumbers \LineNumbersfalse %
\def\nolinenumbers{}
\def\linenumbers{}
\def\linenumbers{}
\def\setpagewiselinenumbers{}
\def\linelabel#1{}
\def\lineref#1{}
\def\modulolinenumbers#1{}
}

\usepackage[bookmarks,breaklinks,pdfpagelabels,
colorlinks=true,  % Colors links instead of ugly boxes
linkcolor=black,  % Color of internal links
]{hyperref}[2012/11/06] % Reference package, must be before cleveref

\usepackage[noabbrev,%
nameinlink
]{cleveref}[2013/12/28]

\usepackage{longtable}
%\usepackage{booktabs}

\usepackage{xparse}
\usepackage{tcolorbox}
\tcbuselibrary{most}
\tcbuselibrary{minted}
\tcbuselibrary{listings}
\tcbuselibrary{breakable}
\tcbuselibrary{skins}
\usetikzlibrary{shadings}

\input{erratadocutil}

\usepackage{tabularx}
\usepackage[strings]{underscore} % Allow underscores in strings and filenames

\usepackage[inline]{enumitem}
\newlist{inlinelist}{enumerate*}{1}
\setlist*[inlinelist]{%
  label*=\textit{\alph*)},
}


\usepackage[
%show=true,
margins=true,
foots=true,
record=true,
linenos=true,
%cleveref=false,
%hyperref=true,
defaultargs={margin=true,foot=false},
%defaultargsadd={margin=false},
%defaultargsdelete={margin=false},
requirecmddescription=false,
%marginnote=false,
uniquelistlabels=0,
%inlinechanges=false,
environmenttestmode=2,
inlinemarks=true,
%styles={default, detailed, tablelist},
%styles={detailed},
styles={tablelist},
%styles={default},
]{errata}[2015/04/16]

\errataSetKeyFormat[#1+#2]{marginfmt/marginnote}{%
{\vspace{-4pt}%
\begin{tcolorbox}[size=fbox,nobeforeafter,
    colframe=red!50!white,colback=red!25!yellow!5!white,%
    fontupper=\scriptsize,%
    bottomrule=1pt,toprule=1pt,leftrule=0pt,rightrule=0pt,
%    arc=0cm, outer arc=0mm, % Remove the rounded corners
    arc=0cm,
    outer arc=0mm, % Remove the rounded corners
    code={\pgfkeysalsofrom#1}
    ]#2\end{tcolorbox}}}%

\crefname{paragraph}{\S}{\S\S} % default is {paragraph}{paragraphs}

\setcounter{secnumdepth}{6}

\makeatletter
\newcommand{\getCurrentSectionNumber}{%
  \ifnum\c@section=0 %
  \thechapter
  \else
  \ifnum\c@subsection=0 %
  \thesection
  \else
  \ifnum\c@subsubsection=0 %
  \thesubsection
  \else
  \thesubsubsection
  \fi
  \fi
  \fi
}


% Customize table to show line numbers
\errataSetKeyFormat[#1]{listformats/erratalistbegin}{\makeatletter%
  \begin{longtable}{ p{1cm} c c p{2cm} p{5cm} c c }%
    & Title & Type & Reported by & Action Desc & Page & Line \\\hline%
    }

\errataSetKeyFormat[#1]{listformats/reported-by}{R: #1}
\errataSetKeyFormat[#1]{listformats/lineref}{\lineref{#1}}

\errataSetListFormat{+erratumlistitemtitle,+erratumlisttarget,\noexpand\errataamersand,%
    +erratumnamedlink,\noexpand\errataamersand,+action,\noexpand\errataamersand,%
    -reported-by,\noexpand\errataamersand,%
    +actiondescription,\noexpand\errataamersand,+pageref,\noexpand\errataamersand,-lineref}%

\makeatother

\newlength{\namelen}
\setlength\namelen{39mm}


\NewDocumentCommand{\erratatest}{ m m }
{\noindent {\makebox[\namelen][l]{\textbf{#1}}}: #2\\}



\begin{document}

\chapter{Chapter 1}\label{chap-one}

\immediate\typeout{Erratum BEGIN}%


\erratumAdd[margin=false, foot=true, description={SKROT}]{In par1}

\erratumAdd[margin=false, foot=true, description={add}]{In par1}


ERRAT1 CREF: \cref{errata1}

ERRATA1 CPAGEREF: \cpageref{errata1}

ERR1 LINE NUMBER: \lineref{erratalineno1}.


Therefore, the'
\erratumReplace[id={testid}, type={Correction}, description={grammar}, reported-by=bro]{results}{result}'

\section{Section 1.1}\label{sect-one}

Note test \erratumNote[marginargs={before,after,%
,colframe=blue!50!white}]{Changes to paragraph} space before?


\cref{errata4} LINE NUMBER: \lineref{erratalineno4}.

What's for? Its contents is delivered into the token list par1 \erratumAdd[label={customlabel0}]{added in par1}.

\cref{errata5} (\cref{customlabel0}) LINE NUMBER: \lineref{erratalineno5}.

TeX is reading when it switches from vertical to horizontal mode. \erratumAdd{added in par2} This happens when TeX, in vertical mode, sees a horizontal command par2.

At that time the horizontal command is set aside; \erratumAdd{added in par3} before examining it again par3.



Adding comma here\erratumAdd{,} horizontal command is set aside;

Sect1: \cref{sect-one}  (\getCurrentSectionNumber)

Replace: Next' \erratumReplace{word}{WORD} 'has been replaced.

\cref{errata9} LINE NUMBER: \lineref{erratalineno9}.


%Hyperlink: '\hyperlink{firstreplace}{LINK}'

%\nolinenumbers
Add: A word has been' \erratumAdd[margin=true,record=true]{added} 'here.
%\footnote{TEST}

%\linenumbers

Delete: A word has been' \erratumDelete{removed} 'removed here, and here' \erratumDelete{removed}'.

\erratumAdd[type={refining}, description={Added listing}]{chap-one}
%\erratumAdd[type={refining}, description={Added listing}, margin=true,foot=true,inlinechange=true,inlinemark=true]{CREF INSIDE \cref{chap-one}}



\erratumNote{Changes to paragraph} \erratumNote{Changes to paragraph}

TEST \linelabel{linetestone} TWO \linelabel{linetesttwo} LINELABELS \lineref{linetestone}, \lineref{linetesttwo} \pageref{linetesttwo}


\paragraph{TEST paragrah:}

Delete: A word has been removed

with space:' \erratumDelete{removed} 'and without'\erratumDelete{removed}'.

 \erratumDelete{removed} at beginning of sentence.

Delete comma'\erratumDelete{,} 'prev space is removed.

Removed a word'\erratumDelete{removed} 'with no previous space.



SECT1: \cref{sect-one}
SECT1PREF \cpageref{sect-one}


\erratumReplace[id={testid}, description={grammar}]{tulls}{tull}
,\erratumAdd[id={testid}, type={Addition}, date={2015/03/18}]{added},


\subsection{Subsection 1.1}\label{subsect-one}



SUBSECT1.1: \cref{subsect-one} (\getCurrentSectionNumber)


\erratumAdd[inlinemark=false]{noinline}


\erratumReplace[type={grammar}, description={Fix gramar}]{in}{of}

\subsubsection{Subsubsection 1.1.1}\label{subsubsect-one}

\erratumAdd[id={testid}, description={grammar}]{insubsubsect}

SUBSUBSECT1: \cref{subsubsect-one} (\getCurrentSectionNumber)

\paragraph{Paragraph 1}\label{par-one}

\erratumAdd[id={testid}, description={grammar}]{inparagraph}

PAR1: \cref{par-one}  (\getCurrentSectionNumber)

%ERR1: \cref{errata1}

ERR1 ref: \ref{errata1}




\subsubsection{Subsubsection 1.1.2}\label{subsubsect-two}

\erratumAdd[id={testid}, description={grammar}]{insubsubsecttwo}

SUBSUBSECT2: \cref{subsubsect-two} (\getCurrentSectionNumber)


\subsection{Subsection 1.2}\label{subsect-two}

SECT2: \cref{subsect-two}

\erratumAdd[id={testid}, description={grammar}]{insubsect2}



\paragraph*{Paragraph 2}\label{par-two}

\erratumAdd[inlinemark=false]{noinline}


\erratumAdd[inlinemark=false]{noinline}

\section{Section 2}\label{sect-two}

\subsection{Subsection 2.1}\label{subsect-two-one}

\chapter{Chapter2}\label{chap-two}


HELLO \erratumAdd{In chap2 par1.}

PAR1: \cref{chap-two}  (\getCurrentSectionNumber)

\erratumAdd{in chap two}

PAR ONE: \cref{par-one}

SECT1: \cref{subsect-one}


\section{Section 2.1}\label{sect-two-one}



SECT2.1: \cref{sect-two-one}  (\getCurrentSectionNumber)

\erratumAdd[inlinemark=true]{inlinemark}

\begin{inlinelist}
  \item{} has the \erratumDelete[type={grammar}]{marg} same behavior, and

  \item{} when the competing traffic does not care about per-packet latency.
\end{inlinelist}


Here we have removed \erratumDelete[id={testid}, type={Deletion}]{removed} a word.

ERRATA35 pref: \pageref{erratalineno35}, lineref: \lineref{erratalineno35}

\erratumAdd{In this experiment, they are limited to bundle one previous segment.}


\begin{erratum}[date=2006-07-19,reported-by=Michael Kohlhase,margin=true,
marginargs={colframe=blue!50!white}]{old should be new}
this is a test of a long erratum
\begin{enumerate}
  \item We can replace \eReplace{oldtext}{newtext}
  \item and \eAdd{new text}
  \item and finally delete old text\eDelete{alltogether}
  \end{enumerate}
\end{erratum}

\subsection{Subsection Margin test}\label{subsect-margin}

\erratatest{}{Add test \erratumAdd{added} sentence test.}
\erratatest{}{Add test \erratumAdd[label={customlabel1}]{added} sentence test.}
\erratatest{inlinemark=false}{Add test \erratumAdd[inlinemark=false]{added} sentence test.}
\erratatest{marginargs={size=fbox}}{Add test \erratumAdd[marginargs={size=fbox,colframe=green}]{added} sentence test.}

customlabel: \Cref{customlabel1}

\erratatest{}{INNER MODE TEST}

ERRAT40: \cref{errata40}

\begin{gather}\label{gatherlabel}
  a=a  \\
  b=b \erratumAdd[uselabel=gatherlabel,foot=true,margin=true]{1}
\end{gather}
%  b=b \erratumAdd[uselabel=gatherlabel]{\text{1}}
%  b=b \erratumAdd[uselabel=gatherlabel]{\text{$1$}}
REF(gatherlabel): \ref{gatherlabel}


\begin{equation}\label{equationlabel}
\begin{split}
  a=a  \\
  b=b \erratumAdd[uselabel=equationlabel,foot=true,margin=true]{equation test}
\end{split}
\end{equation}
%  b=b \erratumAdd[uselabel=equationlabel]{\text{Spaces here should be preserved}}
REF(equationlabel): \ref{equationlabel}


\erratumAdd[uselabel=equationlabel]{\text{Spaces here should be preserved}}

\subsection{Subsection Space tests}\label{subsect-space}

\subsubsection{erratumAdd space test}

\erratatest{inlinemark=false}{Add test \erratumAdd{added} sentence test.}
\erratatest{inlinemark=false}{Add test \erratumAdd[inlinemark=false]{added} sentence test.}
\erratatest{inlinechange=false}{Add test \erratumAdd[inlinechange=false]{added} sentence test.}
\erratatest{inlineboth=false}{Add test \erratumAdd[inlinechange=false, inlinemark=false]{added} sentence test.}
\erratatest{}{Add test\erratumAdd{,} sentence test.}
\erratatest{inlinemark=false}{Add test\erratumAdd[inlinemark=false]{,} sentence test.}


\subsubsection{erratumReplace space test}

\erratatest{inlinemark=false}{Replace test \erratumReplace{orig}{repl} sentence test.}
\erratatest{inlinemark=false}{Replace test \erratumReplace[inlinemark=false]{orig}{repl} sentence test.}
\erratatest{inlinechange=false}{Replace test \erratumReplace[inlinechange=false]{orig}{repl} sentence test.}
\erratatest{inlineboth=false}{Replace test \erratumReplace[inlinechange=false, inlinemark=false]{orig}{repl} sentence test.}
\erratatest{inlinechange=true}{Replace test \erratumReplace[inlinechange=true]{orig}{repl} sentence test.}

\subsubsection{erratumDelete space test}

\erratatest{inlinemark=false}{Delete test \erratumDelete[inlinemark=false]{deleted} sentence test.}
\erratatest{inlinemark=false}{Delete test \erratumDelete{deleted} sentence test.}
\erratatest{inlinechange=false}{Delete test \erratumDelete[inlinechange=false]{deleted} sentence test.}
\erratatest{inlineboth=false}{Delete test \erratumDelete[inlinechange=false, inlinemark=false]{deleted} sentence test.}

\subsubsection{erratumNote space test}

\erratatest{inlinemark=false}{Note test \erratumNote[inlinemark=false]{note1} sentence test.}

\erratatest{inlinemark=false}{Note test \erratumNote{note2} sentence test.}

\erratatest{inlinechange=false}{Note test \erratumNote[inlinechange=false]{note3} sentence test.}
\erratatest{inlineboth=false}{Note test \erratumNote[inlinechange=false, inlinemark=false]{note4} sentence test.}


\chapter{Testing special environments}\label{chap-special-environments}


\subsection{Equation}\label{subsect-Equation}

\begin{examplecode}{codeandresult, title=Title}
\begin{equation}\label{customlabel}
\begin{split}
  a=a  \\
  b=b \erratumAdd[uselabel=customlabel]{custom label example}
  [Space test]
\end{split}
\end{equation}
\end{examplecode}
%\docaption{Some short errata}

\subsection{Tables}\label{subsect-Tables}

\begin{table}[htbp]
  \centering
  \begin{tabularx}{\textwidth}{c|c|c}
    \hline
    Alpha     & Beta     & Gamma \erratumAdd[usemarginnote,margin=true,foot=true]{added} \erratumAdd[usemarginnote=20pt]{added2} \\
    0         & 2        & 4         \\ \hline
    1         & 3        & 5         \\ \hline
  \end{tabularx}
\end{table}
%\immediate\typeout{Erratum AFTER TABULARX}%


\subsection{Colorboxes}\label{subsect-Colorboxes}

\begin{lstlisting}
In this sentence\erratumAdd{.} we added a comma without an
inline mark\erratumAdd[inlinemark=true]{..} and a comma with
inline marking.
\end{lstlisting}



\begin{examplecode}{codeandresult,minted options={},title=Title}
In this sentence\erratumAdd{,} we added a comma without an
inline mark\erratumAdd[inlinemark=true]{,} and a comma with
inline marking.
\end{examplecode}



\begin{examplecode}{latexcode, label={code:short-errata}}
Here we have \erratumAdd{three} errata in one
\erratumDelete{superfluous phrase}{darned} long \erratumReplace
[description={translated}]{Zeile}{line}
\end{examplecode}

\chapter{Errata}\label{chap:errata}

\PrintErrata

\end{document}

%%% Local Variables:
%%% TeX-master: "erratatest"
%%% End:
